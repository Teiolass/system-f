\documentclass[]{marticle}
\usepackage{mstyle}

\title{\textbf{\huge Rappresentabilit\`a di Funzioni nel Lambda-calcolo
Polimorfico}}
\date{}


\begin{document}
\maketitle

\section{Introduzione}

Il $\l$-calcolo \`e un sistema formale sviluppato negli anni '30 da Alonzo
Church. Lo scopo originario era quello di fondare la matematica, e non \`e stato
raggiunto in quanto si rivel\`o inconsistente, come dimostrato da Kleene e Rosser
nel 1936. Un sottoinsieme del sistema si \`e comunque sviluppato per la sua
capacit\`a di esprimere computazioni mediante astrazione su variabili e
sostituzione. Lo studio del $\l$-calcolo \`e dunque lo studio di entit\`a dette
$\l$-termini che svologono allo stesso tempo il ruolo di programmi e di dati su
cui i programmi lavorano. Sui termini si andr\`a a considerare una relazione di
ordine, detta riduzione, che rappresenta l'esecuzione dei programmi. All'interno
del $\l$-calcolo, con opportune codifiche \`e possibile rappresentare i numeri
naturali e le tutte funzioni calcolabili, parziali e totali. Poich\`e l'insieme delle funzioni
calcolabili totali non \`e ricorsivamente enumerabile esistono dei termini la
cui computazione non termina (diremo che non sono normalizzanti).

Il $\l$-calcolo tipato \`e una variante del $\l$-calcolo ad ogni termine \`e
associata un'entit\`a sintattica detta tipo. Esso ha origine nei lavori di
Haskell Curry (1934) e di Church (1940).  La riduzione in questo caso \`e
ridefinita aggiungendo vincoli sul come \`e possibile comporre i termini in base
al loro tipo, e in particolare rende chiara la distinzione tra dati e programmi.
Come conseguenza si pu\`o dimostrare che questa variante ha la propriet\`a di
normalizzazione, ovvero tutte le computazioni terminano e tutti i termini sono
normalizzanti. I tipi sono studiati anche in ambito informatico per la verifica
in modo automatico della presenza di alcuni errori che dovrebbero essere
altrimenti cercati a mano dal programmatore.

In questa tesi ci occuperemo di una variante del calcolo detta sistema F, anche
nota come $\l$-calcolo polimorfico o $\l$-calcolo del secondo ordine. Essa \`e
stata sviluppata dal logico Jean-Yves Girard (1972) e dall'informatico John
Charles Reynolds (1974). Il sistema F \`e essenzialmente una variante del
$\l$-calcolo tipato in cui viene aggiunta una quantificazione universale sui
tipi.

Anche per il sistema F vale la propriet\`a di normalizzazione. Troveremo dunque
che le funzioni rappresentabili nel sistema F sono un sottoinsieme delle
funzioni calcolabili totali, e ne daremo una caratterizzazione pi\`u precisa:
esse sono esattamente le funzioni di cui l'aritmetica di Heyting del secondo
ordine dimostra la totalit\`a.  Mostreremo quindi un esempio di funzione non
rappresentabile nel sistema F e dedurremo dunque la consistenza dell'aritmetica
di Peano del secondo ordine a partire dal risultato di normalizzazione.

Metteremo inoltre tale risultato in confronto con un risultato equivalente su un
altra variante del $\l$-calcolo, detta sistema T di G\"odel, per cui vale
ugualmente la propriet\`a di normalizzazione. Nel sistema T infatti le funzioni
rappresentabili sono esattamente quelle che l'aritmetica di Heyting del primo
ordine dimostra essere totali.

\section{Il $\lambda$-calcolo non tipato}

Il $\l$-calcolo \`e un sistema formale per descrivere delle computazioni.
[\dots]

In questa tesi ci occuperemo di studiare alcune propriet\`a di diversi tipi
di $\lambda$-calcoli. Iniziamo dunque definendo gli elementi fondamentali del
$\lambda$-calcolo pi\`u semplice, ovvero quello non tipato.

\begin{block}[Definizione]
    I termini del $\l$-calcolo semplice si definiscono induttivamente come:
    \begin{itemize}
        \item Le variabili $x_1, x_2, \dots$ sono termini.
        \item Se $t$ e $v$ sono termini, allora anche l'applicazione $(tv)$ \`e
            un termine.
        \item Se $t$ \`e un termine e $x$ \`e una variabile, l'astrazione $(\l
            x.  t)$ \`e un termine.
    \end{itemize}

    Dato un termine $t$, i sottotermini sono tutti i termini che appaiono nella
    costruzione induttiva di $t$.
\end{block}

L'idea intuitiva dietro questa definizione \`e quella che un termine del tipo
$\l x.t$ corrisponde a un programma con input $x$ e corpo $t$. Inoltre
un'applicazione della forma $tv$ rappresenta  un programma $t$ quando con input
$v$. Un fatto interessante da notare \`e che i termini possono svolgere
indistintamente il ruolo di programma e di dato su cui un programma opera.

Per comodit\`a e leggibilit\`a delle notazioni, ometteremo spesso le parentesi
sottointendendo che l'applicazione si associa a sinistra (e quindi $xyz=(xy)z$)
e l'astrazione si associa a destra, usando un singolo simbolo $\l$, per esempio
$\l xyz.yyxz = \l x. (\l y. (\l z. yyxz))$. \deb{Vettori di variabili.}

In modo simile a quanto si fa comunemente in logica \`e utile distinguere le
occorrenze di una variabile in una formula tra occorrenze libere e legate. In
particolare l'astrazione su una varibile lega tale variabile. Pi\`u formalmente:

\begin{block}[Definizione]
    Un'occorrenza della variabile $x$ in un termine $t$ si dice legata se esiste
    un sottotermine del tipo $\l x. t'$ che la contiene. Si dice legata
    altrimenti.

    Inoltre, dato un termine $t$ si definiscono induttivamente le sue variabili
    libere come:
    \begin{itemize}
        \item Se $t=x$ dove $x$ \`e una variabile, allora \`e l'unica variabile
            libera d $t$.
        \item Se $t=uv$, allora le variabili libere di $t$ sono tutte e sole le
            variabili libere che compaiono in $u$ o in $v$.
        \item Se $t=\l x. u$, allora le variabili libere di $t$ sono tutte le
            variabili libere di $u$ con l'esclusione di $x$.
    \end{itemize}
\end{block}

Nel seguito considereremo i termini modulo il rinominare le variabili legate.
Questo corrisponde al fatto che in un programma \`e possibile rinominare i
parametri formali delle funzioni (modificando consistentemente le loro
occorrenze nei corpi di tali funzioni). In particolare la relazione di
equivalenza che mette in relazione un termine con tutti i termini uguali a esso
a meno del rinominare le variabili legate si chiama $\alpha$-equivalenza. Da qui
in avanti, per semplicit\`a, abuseremo della notazione riferendoci alle classi
di equivalenza con i loro elementi.

Con qualche accortezza per evitare la cattura delle variabili si pu\`o definire
la sostituzione di un termine su una variabile.

\begin{block}[Definizione]
    Se $u$ e $v$ sono termini e $x$ \`e una variabile, allora
    la sostituzione di $v$ su $x$ in $u$ \`e il termine $u \sub{v}{x}$ definito
    come:
    \begin{itemize}
        \item Se $u=x$ allora $u \sub{v}{x} = v$.
        \item Se $u=y$ con $y$ una variabile distinta da $x$, allora $u
            \sub{v}{x} = y$.
        \item Se $u = tw$, allora $u \sub{v}{x} = (t \sub{v}{x})(w \sub{v}{x})
            $.
        \item Se $u = \l y. t$ e $y$ \`e distinta da $x$ e non appare libera in
            $v$, allora $u \sub{v}{x} = \l y. t\sub{v}{x}$.
    \end{itemize}
\end{block}

A questo punto abbiamo presentato tutti gli strumenti per introdurre una
relazione fondamentale sui termini del calcolo, ovvero la conversione.

\begin{block}[Definizione]
    Dati due termini $u$ e $v$, si dice che $u$ si converte a $v$ e scriverermo
    $u \conv_C v$, se $v$ \`e ottenuto da $u$ sostituendo un sottotermine nella
    forma $(\l x. u')u''$ con $u'\sub{u''}{x}$ (quando tale sostituzione \`e
    permessa e quindi non si ha cattura delle variabili).

    Si dice che $u$ si riduce a $v$ se esistono $n$ termini $u_1, \dots, u_n$
    con $u_1 = u$ e $u_n = v$ tali che $u_1 \conv_C u_2 \conv_C \dots \conv_C
    u_n$. In tal caso scriveremo $u\conv v$.
\end{block}

Notiamo che la definizione dice essenzialmente che $\conv$ \`e la chiusura
transitiva di $\conv_C$. Tali nozioni di conversione e riduzione sono chiamati
in letteratura anche $\beta$-conversione e $\beta$-riduzione. La pi\`u piccola
relazione di equivalenza che contiene la $\beta$-conversione \`e detta
$\beta$-equivalenza.

L'idea di conversione corrisponde all'esecuzione di un passo del programma
corrispondente al termine che viene convertito. Notiamo per\`o che volendo
ridurre un termine, la successione delle conversioni non \`e univocamente
determinata, come nel termine $((\l x .xx)y)((\l x.x)z)$.

Facciamo ora due esempi importanti di riduzione.

\begin{block}[Esempio]
    \begin{gather*}
        (\l f . \l x . f(fx))(\l y. yy) \conv_C  \\
        \l x. (\l y. yy)((\l y.yy) x) \conv_C \\
        \l x .(\l y. yy)(xx) \conv_C \\
        \l x. xxxx.
    \end{gather*}
\end{block}

\begin{block}[Esempio]
    \begin{gather*}
        \Omega = (\l x.xx)(\l x.xx) \conv_C \\ 
        (\l x.xx)(\l x.xx) \conv_C \\
        \dots
    \end{gather*}
\end{block}

Notiamo che il primo esempio finisce con un termine che non pu\`o essere
ulteriormente convertito, mentre nel secondo la successione di conversioni \`e
infinita. Questa distinzione \`e importante e conduce alle seguenti definizioni.

\begin{block}[Definizione]
    Un termine si dice in forma normale se non pu\`o essere ulteriormente
    convertito. Un termine si dice normalizzante se pu\`o essere convertito a un
    termine in forma normale.
\end{block}

Esistono comunque termini che pur essendo normalizzanti ammettono una
successione infinita di conversioni:

\begin{block}[Esempio]
    \begin{gather*}
        (\l x . \l y. y)\Omega z \conv_C z
    \end{gather*}
    ma convertendo ad ogni passo il termine $\Omega$ (si veda la sua definizione
    nell'esempio precedente), si ottiene che:
    \begin{gather*}
        (\l x . \l y. y)\Omega z \conv_C \\ 
        (\l x . \l y. y)\Omega z \conv_C \\
        \dots
    \end{gather*}
\end{block}

\`E allora utile introdurre le seguenti nozioni:
\begin{block}[Definizione]
    Dato un termine $t$ si definisce $\nu(t)$ il massimo numero di conversioni
    necessarie per portare $t$ in forma normale, ossia
    \begin{gather*}
        \nu (t) = \sup \setof { n \st
            \exists u_1, \dots, u_n \text{ per cui } \\
            t \conv_C u_1 \conv_C \dots
            \conv_C u_n \text{ e $u_n$ \`e in forma normale}
        }.
    \end{gather*}

    Un termine $t$ si dice fortemente normalizzante se $\nu(t) < \infty$.
\end{block}

Si osservi che se un termine \`e fortemente normalizzante allora \`e anche
normalizzante, e non pu\`o essere convertito un numero infinito di volte.

\deb{Convertendo sempre la redex pi\`u a sinistra si ottiene comunque sempre una
forma normale.}

Un'importante propriet\`a di cui gode la riduzione \`e detta propriet\`a di
Church-Rosser, che in particolare implica l'unicit\`a del termine in forma
normale a cui si converte un termine normalizzante.

\begin{block}[Teorema]
    Sia $t$ un termine. Se $u$ e $v$ sono termini per cui $t\conv u$ e $t\conv
    v$, allora esiste un quarto termine $w$ tale che $u\conv w$ e $v\conv w$.
    \deb{Disegno del diamante.}
\end{block}

\subsection{Rappresentabilit\`a nel $\lambda$-calcolo non tipato}

Tra i termini del $\l$-calcolo ne possiamo individuare alcuni per metterli in
corrispondenza con i numeri naturali.

\begin{block}[Definizione]
    Dato un numero naturale $n$ definiamo il corrispondente numerale
    $\overline{n}$ come il termine $\l f x. f^n x$, dove il simbolo $f^n x$
    indica $f(f(\cdots f(x) \cdots))$.
\end{block}

In particolare $\overline{n}$ \`e un termine che presa una funzione $f$, la
compone con se stessa $n$ volte. Osserviamo che inoltre i numerali sono termini
in forma normale, dunque per la propriet\`a di Church-Rosser sono anche termini
distinti anche modulo la $\beta$-equivalenza.

\begin{block}[Definizione]
    Data una funzione (eventualmente parziale) $\phi\colon \N \rightarrow \N$,
    si dice che un termine $t$ rappresenta $\phi$ se per ogni coppia di naturali
    $m$ e $n$ vale che $\phi(n)=m$ se e solo se $t\overline{n}\conv\overline{m}$
    e $t\overline{n}$ non \`e normalizzante quando $\phi(n)=\bot$.
\end{block}

Facciamo ora alcuni esempi di funzioni rappresentabili:
\begin{block}[Esempio]
    Il termine $A = \l pq fx. (pf)(qfx)$ rappresenta l'addizione. Per esempio:
    \begin{gather*}
        A\overline{2}\ \overline{3} \conv \l fx. f^2 (f^3 x) = \\
        \l fx. f^5 x = \overline{5}.
    \end{gather*}

    In modo simile esistono i termini $M = \l p q f x. q (p f) x$ e $E = \l p q
    f x. qpfx$ che rappresentano rispettivamente la moltiplicazione e
    l'esponenziazione. Questo ultimo termine \`e leggermente differente dagli
    altri due in quanto \`e l'unico in cui un numerale viene direttamente
    applicato ad un altro numerale. Vedremo che questa differenza sar\`a
    decisiva per la rappresentabilit\`a in alcune varianti di $\l$-calcolo.
\end{block}

\begin{block}[Teorema]
    Le funzioni rappresentabili nel $\l$-calcolo non tipato sono esattamente le
    funzioni calcolabili.
\end{block}

Un nodo cruciale nella dimostrazione del precedente teorema \`e l'esistenza di
un combinatore di punto fisso, ovvero di un termine $t$ tale che per ogni
termine $u$ vale che $tu = u(tu)$, modula la $\beta$-conversione. Un esempio di
combinatore di punto fisso \`e il termine $Y = \l f. (\l x. f(xx))(\l x.
f(xx))$, come si pu\`o facilmente verificare. Esso \`e noto come combinatore di
punto fisso di Curry. Notiamo inoltre che tale termine non \`e normalizzante, ma
pi\`u in generale possiamo dimostrare che nessun combinatore di punto fisso $t$
pu\`o essere normalizzante.  Infatti se indichiamo con $t'$ la forma normale di
$tx$, dove $x$ \`e una variabile, allora anche $t'$ \`e normalizzante. Inoltre
vale che $t' = xt'$, modulo la $\beta$-conversione, ma entrambi questi termini
sono in forma normale, e dunque abbiamo l'assurdo.

\section{Il $\l$-calcolo tipato semplice}

Nel $\l$-calcolo non tipato, esistono termini per cui la normalizzazione non
corrisponde all'idea intuitiva di ``semplificazione'', come
nell'esempio \deb{quello con $\Omega$} oppure nel caso ancora peggiore seguente.

\begin{block}[Esempio]
    \begin{gather*}
        (\l x. xxx)(\l x. xxx) \conv \\
        (\l x. xxx)(\l x. xxx)(\l x. xxx)\conv \\
        \cdots
    \end{gather*}
\end{block} 

Il problema alla base di questo comportamento \`e il fatto che non vi \`e
distizione tra dati e porgrammi e in particolare \`e permessa l'applicazione di
un termine a se stesso. Versioni pi\`u sofisticate del calcolo puntano dunque a
introdurre dei vincoli sull'applicazione dei termini, ed \`e per questo motivo
che si introducono i tipi. L'idea \`e quella di associare ad ogni termine un
tipo, e permettere l'applicazione di termini solo se i loro tipi sono
compatibili.

Definiamo allora una variante del $\l$-calcolo detta $\l$-calcolo tipato
semplice.

\begin{block}[Definizione]
    I tipi del $\l|$-calcolo tipato semplice sono definiti induttivamente come:
    \begin{itemize}
        \item $U_1, U_2, \dots$ sono tipi, detti variabili di tipo e sono anche
            detti tipi atomici.
        \item Se $U$ e $V$ sono tipi, allora anche $(U\rar V)$ \`e un tipo.
    \end{itemize}
    Per comodit\`a, in assenza di parentesi, l'associativit\`a di $\rar$ \`e a
    dstra: per esempio $U\rar V\rar W = (U\rar(V\rar W))$.

    A questo punto ricostruiamo i termini associando a ciascuno di essi un
    relativo tipo.
    \begin{itemize}
        \item Per ogni tipo $U$, le variabili $x^U_1, x^U_2, \dots$ sono termini
            di tipo $U$.
        \item Se $t$ e $v$ sono termini di tipo rispettivamente $U\rar V$ e $U$,
            allora l'applicazione $(tv)$ \`e un termine di tipo $V$.
        \item Se $t$ \`e un termine di tipo $V$ e $x$ \`e una variabile di tipo
            $U$, allora l'astrazione $(\l x.  t)$ \`e un termine di tipo $U\rar
            V$.
    \end{itemize}

    Per indicare che un termine $t$ \`e di tipo $U$ scriveremo anche $t^U$.
\end{block}

Si vede dunque che il tipo $U\rightarrow V$ corrisponde al tipo delle funzioni
dai termini di tipo $U$ ai termini di tipo $V$, e che l'applicazione \`e
consentita solo quando ``il dominio della funzione e il tipo dell'argomento
coincidono''.

In modo identico a quanto gi\`a fatto per il $\l$-calcolo semplice, \`e
possibile definire le nozioni di conversione, riduzione e forma normale. Si noti
che entrambe le relazione conservano il tipo dei termini.

\subsection{Normalizzazione Forte per il Tipato Semplice}
L'obiettivo di questa sezione \`e quella di dimostrare il seguente importante
risultato:
\begin{block}[Teorema]
    Tutti i termini del $\l$-calcolo tipato semplice sono fortemente
    normalizzanti.
\end{block}

Da questo fatto discende che l'espressivit\`a di questo calcolo \`e molto
ridotta rispetto a quella del $\l$-calcolo tipato semplice. Per esempio non
possiamo trovare nessun combinatore di punto fisso e vedremo che la classe di
funzioni rappresentabili \`e anch'essa ridotta.

Definiamo come prima cosa la nozione di riducibilit\`a. 
\begin{block}[Definizione]
    Sia $U$ un tipo, e $t$ un termine di tipo $U$. Definiamo induttivamente il
    suo insieme di riducibilit\`a $\red{U}$ come:
    \begin{itemize}
        \item Se $U$ \`e atomico, $t$ \`e riducibile se e solo se \`e fortemente
            normalizzante.
        \item Se $T = V \rightarrow W$, $t$ \`e riducibile se e solo se per ogni
            termine riducibile $v$ di tipo $V$, il termine $tv$ \`e riducibile
            di tipo $W$.
    \end{itemize}
\end{block}

Osserviamo che sono termini riducibili tutte le variabili di tipo.

\begin{block}[Definizione]
    Diciamo che un termine $t$ \`e neutrale se \`e nella forma $xy$.
\end{block}

L'idea dietro alla neutralit\`a \`e che se $t$ \`e un termine neutrale e $v$ un
termine per cui $tv \conv_C u$, allora $u = t' v$ oppure $u=tv'$ dove $t'$ e $v'$
sono conversioni rispettivamente di $t$ e $v$. In particolare non ci sono
step di riduzione in cui $v$ o un suo sottotermine viene sostituito in una
variabile di $t$.

Dimostriamo ora che gli insiemi di riducibili godono di alcune propriet\`a, che
saranno utili a dimostrare il teorema di questa sezione e torneranno anche utili
nello studio della normalizzazione nel sistema F.

\begin{block}[Proposizione]
    \begin{nlist}[CR1]
        \item Se $t\in \red{U}$, allora $t$ \`e fortemente normalizzante.
        \item Se $t\in \red{U}$ e $t\conv u$, allora $u\in\red{U}$.
        \item Se $t$ \`e neutrale di tipo $U$ e per ogni $t'$ per cui $t\conv_C
            t'$ vale che $t'\in\red{U}$, allora anche $t\in\red{U}$.
    \end{nlist}
\end{block}

Notiamo che la prima propriet\`a indica che essere riducibile implica l'essere
fortemente normalizzante. La seconda propriet\`a permette di conoscere la
riducibilit\`a di un termine data la riducibilit\`a di un termine precedente in
una catena di conversioni. Infine la terza propriet\`a permette di conoscere la
riducibilit\`a di un termne data quella delle sue conversioni.

\begin{proof}
    La dimostrazione \`e per induzione sulla complessit\`a dei tipi.

    Iniziamo dal caso in cui il tipo $U$ sia una variabile di tipo. Allora,
    poich\`e i riducibili di tipo $U$ sono i termini fortemente normalizzanti,
    (CR1) \`e una tautologia. Se un termine $t$ \`e fortemente normalizzante e
    $t\conv t'$, allora anche $t'$ \`e fortemente normalizzante perch\'e vale
    che $\nu(t')<\nu(t)$. Dunque anche (CR2) vale. Per (CR3), sia $t$ un termine
    neutrale per cui tutte le conversioni sono fortemente normalizzanti. Allora
    vale che $\nu(t)$ \`e pari al massimo di $\nu(t')$ al variare di $t'$ tra le
    conversioni di $t$, e dunque \`e finito.

    Consideriamo adesso il tipo $U\rar V$.  Supponiamo che $t$ sia un riducibile
    di tale tipo, e supponiamo che $x$ sia una variabie d tipo $U$. Poich\`e $x$
    \`e neutrale e normale, essa \`e riducibile. Allora anche $tx$ \`e
    riducibile, per la definizione di riducibilit\`a sul tipo freccia.
    Osserviamo ora che $\nu(t)\leq \nu(tx)$, infatti ad ogni catena di
    conversioni $t\conv_C t_1 \conv\cdots \conv t_n$ possiamo associare la
    catena $tx \conv_C t_1 x \conv\cdots \conv t_n x$. Poich\`e $\nu(tx)$ \`e
    finito, $t$ \`e fortemente normalizzante, e (CR1) \`e dimostrato.

    Se consideriamo un termine $t$ di tipo $U\rar V$ riducibile e un termine
    $t'$ tale che $t\conv t'$, allora per ogni termnine $u$ di tipo $U$ vale che
    $tu\conv t'u$. Utilizzando l'ipotesi induttiva di (CR2) su $V$, otteniamo
    che anche $t'u$ \`e riducibile. Per cui anche $t'$ \`e riducibile e (CR2)
    vale.

    Supponiamo ora di avere $t$ neutrale per cui tutte le sue conversioni siano
    riducibili. Sia $u$ un riducibile di tipo $U$. L'obiettivo \`e mostrare che
    $tu$ \`e riducibile. Per l'ipotesi induttiva per $U$, gi\`a sappiamo che $u$
    \`e fortemente normalizzante, per cui possiamo ragionare per induzione su
    $\nu(u)$. Notiamo che per neutralit\`a di $t$, $tu$ si pu\`o convertire
    soltanto in $t'u$, con $t'$ conversione di $t$, oppure in $tu'$, con $u'$
    conversione di $u$. Nel primo caso sappiamo che $t'$ \`e riducibile, e
    dunque anche $t'u$ lo \`e. Nel secondo caso possiamo osservare che $\nu(u')
    < \nu(u)$ e dunque per induzione otteniamo nuovamente che $t'u$ \`e
    riducibile. Poich\`e $tu$ si converte soltanto a riducibili, \`e anch'esso
    riducibile, per ipotesi induttiva di (CR3). 
\end{proof}

A questo punto dimostriamo un utile lemma.

\begin{block}[Lemma]
    Se per tutti i termini riducibili $u$ di tipo $U$, il termine $v\sub{u}{x}$
    \`e riducibile, allora anche il termine $\l x.v$ \`e riducibile.
\end{block}

\begin{proof}
    Supponiamo che $v\sub{u}{x}$ sia di tipo $V$, allora il termine $\l x.v$ \`e
    di tipo $U\rar V$. Allora vogliamo dimostrare che per ogni termine
    riducibile $u$ di tipo $U$ vale che $(\l x.v)u$ \`e riducibile. Notiamo che
    $v$ \`e riducibile, infatti $x$ \`e riducibile di tipo $U$.

    Ragioniamo per induzione sulla somma $\nu(v)+\nu(u)$ per dimostrare che
    tutte le conversoni di $(\l x.v)u$ sono riducibili. Il termine $(\l x.v)u$
    si pu\`o convertire in:
    \begin{itemize}
        \item $v\sub{u}{x}$, che \`e riducibile per ipotesi.
        \item $(\l x. v')u$, con $v'$ conversione di $v$. Allora si ha che $v'$
            \`e riducibile per (CR2), e vale $\nu(v')<\nu(v)$ e quindi per
            ipotesi induttiva $\l x. v'$ \`e riducibile.
        \item $(\l x. v)u'$, con $u'$ conversione di $u$. In questo caso,
            similmente a prima, $u'$ \`e riducibile, e vale $\nu(v')<\nu(v)$.
            Nuovamente $(\l x. v)u'$ \`e anch'esso riducibile per ipotesi
            induttiva.
    \end{itemize}
    Concludiamo per (CR3), che assicura che $\l x.v$ sia dunque riducibile.
\end{proof}

Adesso si dimostra una versione pi\`u forte del teorema.

\begin{block}[Proposizione]
    Sia $t$ un termine le cui variabili libere compaiono tra $x_1, \dots, x_n =
    \underline{x}$, di tipo rispettivamente $U_1, \dots, U_n$. Siano $u_1,
    \dots, u_n = \underline{u}$ termini riducibili di tipo rispettivamente $U_1,
    \dots, U_n$. Allora il termine $t\sub{\underline{u}}{\underline{x}}$ \`e
    riducibile. Intediamo per $t\sub{\underline{u}}{\underline{x}}$ la
    sostituzione $t\sub{u_1}P{x_1}\cdots\sub{u_n}P{x_n}$.
\end{block}

\begin{proof}
    Per induzione sulla complessit\`a di $t$:
    \begin{itemize}
        \item Se $t=x_i$, allora la tesi \`e una tautologia.
        \item Se $t=wv$, allora per l'ipotesi induttiva $w\sub{\underline{u}}
            {\underline{x}}$ e $v\sub{\underline{u}} {\underline{x}}$ sono
            riducibili. Ne consegue che $t\sub{\underline{u}}{\underline{x}} =
            w\sub{\underline{u}} {\underline{x}} v\sub{\underline{u}}
            {\underline{x}}$ \`e riducibile.
        \item Se $t=\l y. w$ di tipo $V\rar W$, allora per ipotesi induttiva,
            $t\sub{\underline{u}} {\underline{x}} \sub{v}{y}$ \`e riducibile per
            tutti i termini $v$ di tipo $V$. Allora per il lemma \deb{quale
            lemma?} si ottiene che $\l y. w\sub{\underline{u}}{\underline{x}}$
            \`e riducibile.
    \end{itemize}
\end{proof}

La dimostrazione del teorema di normalizzazione forte segue da quella della
proposizione ponendo $\underline{u} = \underline{x}$.


\subsection{Rappresentabilit\`a per il Tipato Semplice}
All'interno del $\l$-calcolo tipato semplice \`e possibile rappresentare i
numerali, con gli stessi termini presentati per il caso non tipato. Tuttavia, la
scelta del tipo per in numeri naturali non \`e unica. Infatti per ogni tipo $U$
possiamo costruire per ogni naturale $n$ il corrispondente numerale
\[
    \overline{n} = \l f^{U\rar U} .\l x^U. f^n x
\]
di tipo $\tint = (U\rar U) \rar U \rar U$. Avendo a disposizione i numerali,
possiamo definire le funzioni rappresentabili in modo identico a quanto fatto
con il $\l$-calcolo tipato semplice.

Possiamo associare dei tipi anche ai termini che avevamo usato nel $\l$-calcolo
semplice per rappresentare la somma e la moltiplicazione, infatti 
\[
    A = \l p^\tint q^\tint f^{U\rar U}x^U. (pf)(qfx)
\]
\`e la versione tipata per l'addizione e 
\[
    M = \l p^\tint q^\tint f^{U\rar U}x^U. q (p f) x
\]
lo \`e per la moltiplicazione.

Possiamo dare un ulteriore esempio di funzione rappresentabile, che \`e quella
corrispondente all'\textit{if/then/else}, ovvero la funzione condizionale
$f(x,y,z)$ che vale $y$ se $x$ \`e non nullo e vale $z$ altrimenti. Essa \`e
rappresentata dal termine
\[
    C = \l p^\tint q^\tint r^\tint f^{U\rar U} x^U. p (\l y^U. qfx)(rfx)
\]

Come importante conseguenza della propriet\`a di normalizzazione forte si ha una
conseguente riduzione della classe delle funzioni rappresentabili, che devono
essere per forza totali. Inoltre non tutte le funzioni totali sono
rappresentabili, infatti se cosi fosse sarebbe possibile trovare una loro
enumerazione con i termini del calcolo, ma ci\`o non \`e possibile perch\'e
l'insieme delle (codifiche delle) funzioni totali non \`e ricorsivamente
enumerabile.

Per esempio il termine che avevamo usato per rappresentare l'esponenziazione non
pu\`o essere tipato infatti l'applicazione di un numerale a un altro numerale,
ovvero un termine di tipo $\tint$ a un altro termine di tipo $\tint$, non \`e
permessa dalle regole del $\l$-calcolo tipato semplice.

Si pu\`o dire di pi\`u, perch\'e nessun termine pu\`o rappresentare
l'esponenziazione e pi\`u in generale vale il seguente teorema:

\begin{block}[Teorema]
    Le funzioni rappresentabili nel $\l$-calcolo tipato semplice sono
    esattamente le funzioni generate dalle costanti $0$ e $1$ e dalle funzioni
    di somma, moltiplicazione e condizionale.
\end{block}

Un verso \`e immediato, avendo gi\`a trovato i termini $A$, $M$ e $C$.
\deb{L'altro \`e da fare?}

\section{Il Sistema T di G\"odel}

Il grosso problema del calcolo tipato semplice \`e che i numerali non hanno un
tipo canonico, e non \`e possibile definire funzioni per ricorsione primitiva.
Per ovviare a questo problema introduciamo un nuovo calcolo, il sistema T di
G\"odel, in cui vengono artificialmente inseriti tipi per gli interi e alcuni
termini che rappresentano delle funzioni basilari su di essi. Vedremo che con
queste semplici aggiunte il calcolo guadagna una forte potenza espressiva.

Per definire dunque il sistema T, estendiamo la definizione dei tipi e dei
termnini del $\l$-calcolo tipato semplice. 
\begin{block}[Definizione]
    I tipi del sistema T sono definiti induttivamente come:
    \begin{itemize}
        \item $U_1, U_2, \dots$ sono tipi, detti variabili di tipo.
        \item Se $U$ e $V$ sono tipi, allora anche $(U\rar V)$ \`e un tipo.
        \item Se $U$ e $V$ sono tipi, allora anche $(U\times V)$ \`e un tipo.
        \item $\tint$ \`e un tipo atomico.
    \end{itemize}

    I termini del sistema T sono definiti induttivamente come:
    \begin{itemize}
        \item Per ogni tipo $U$, le variabili $x^U_1, x^U_2, \dots$ sono termini
            di tipo $U$.
        \item Se $t$ e $v$ sono termini di tipo rispettivamente $U\rar V$ e $U$,
            allora l'applicazione $(tv)$ \`e un termine di tipo $V$.
        \item Se $t$ \`e un termine di tipo $V$ e $x$ \`e una variabile di tipo
            $U$, allora l'astrazione $(\l x.  t)$ \`e un termine di tipo $U\rar
            V$.
        \item Se $u$ e $v$ sono termini di tipo rispettivamente $U$ e $V$,
            allora $\ang{u,v}$ \`e un termine di tipo $U\times V$.
        \item Se $t$ \`e un termine di tipo $U\times V$, allora $\pi^1 t$ e
            $\pi^2t$ sono termini di tipo rispettivamente $U$ e $V$.
        \item $O$ \`e un termine di tipo $\tint$.
        \item Se $t$ \`e un termine di tipo $\tint$, allora anche $St$ \`e un
            termine di tipo $\tint$.
        \item Se $u$, $v$, e $t$ sono termini rispettivamente di tipo $U$,
            $U\rar (\tint\rar U)$ e $\tint$, allora $Ruvt$ \`e un termine di
            tipo $U$.
    \end{itemize}
\end{block}

Il significato che vorremmo assegnare ai nuovi termini \`e il seguente: 
il tipo $U\times V$ corrisponde al tipo di una coppia formata da un termine $u$
di tipo $U$ e un termine $v$ di tipo $V$ in quest'ordine. La coppia \`e
rappresentata dal termine $\ang{u,v}$; $\pi^1$ e $\pi^1$ sono le proiezioni.
Il termine $O$ e la struttura $S$ rappresentano rispettivamente lo zero e la
funzione di successore, e pertanto scriveremo il numerale relativo a $n$ come
$S^nO$. Inoltre $R$ \`e l'operatore di ricorsione primitiva, per cui $Ruv(St) =
v(Ruvt)t$.

Con l'introduzione dei nuovi termini occore estendere le regole per la
conversione:
\begin{block}[Definizione]
    Un termine $u$ si converte a $v$ quando $v$ \`e ottenuto sostituendo in $u$
    un sottotermine $u'$ con un termine $v'$ tali che valga una delle seguenti:
    \begin{itemize}
        \item $u'=\pi^1\ang{w,t}$ e $v'=w$, per opportuni termini $w$ e $t$.
        \item $u'=\pi^2\ang{w,t}$ e $v'=t$, per opportuni termini $w$ e $t$.
        \item $u'= (\l x. w)t$ e $v' = w\sub{w}{x}$ dove $x$ \`e una variabile e
            $w$ e $t$ sono termini tali che la sostituzione $w\sub{w}{x}$ sia
            permessa.
        \item $u' = RwtO$ e $v'=w$, per opportuni termini $w$ e $t$.
        \item $u' = Rwt(Sn)$ e $v' = t(Rwtn)n$, per opportuni termini $w$, $t$ e
            $n$.
    \end{itemize}
\end{block}

Come negli altri calcoli si possono definire a questo punto la riduzione, la
forma normale, e la rappresentabilit\`a.

Come osservazione, notiamo che in generale non \`e vero che il termine
$\ang{\pi^1t, \pi^2t}$ si riduce al termine $t$, e anzi potrebbero essere
entrambi in forma normale e pertanto nemmeno $\beta$-equivalenti. \deb{Qui ho
usato Church-Rosser.}

\subsection{Normalizzazione Forte per il Sistema T}

Anche per il sistema T, come per il $\l$-calcolo tipato semplice, vale la
propriet\`a di normalizzazione forte. Lo scopo di questa sezione \`e appunto
quella di dimostrare il seguente teorema:
\begin{block}[Teorema]
    Tutti i termini del sistema T sono fortemente normalizzanti.
\end{block}

La dimostrazione \`e un adattamento di quella gi\`a vista per il tipato
semplice. Estendiamo dunque la definizione di termine riducibile e di termine
neutrale:
\begin{block}[Definizione]
    Sia $U$ un tipo, e $t$ un termine di tipo $U$. Definiamo induttivamente il
    suo insieme di riducibilit\`a $\red{U}$ come:
    \begin{itemize}
        \item Se $U$ \`e atomico, $t$ \`e riducibile se e solo se \`e fortemente
            normalizzante.
        \item Se $T = V \rightarrow W$, $t$ \`e riducibile se e solo se per ogni
            termine riducibile $v$ di tipo $V$, il termine $tv$ \`e riducibile
            di tipo $W$.
        \item Se $T = V \times W$, $t$ \`e riducibile se e solo se sia $\pi^1
            t$ che $\pi^2 t$ sono riducibili.
    \end{itemize}
\end{block}

\begin{block}[Definiizone]
    Un termine si dice neutrale quando non \`e in nessuna delle seguenti forme:
    $\ang{u,v}$, $\l x. v$, $O$ o $S t$.
\end{block}

Con queste definizioni estese possiamo riproporre senza modifiche le propriet\`a
(CR1-3). Continua a valere la proposizione \deb{3.6??}, di cui dobbiamo
estendere la dimostrazione.

\begin{block}[Proposizione]
    Per tutti i riducibili valgono le propriet\`a (CR1-3).
\end{block}

\begin{proof}
    Nonostante sia stato inserito un ulteriore tipo atomico, la dimostrazione
    fatta per essi continua a rimanere valida.

    Studiamo allora il tipo prodotto. Supponiamo che il termine $t$ di tipo
    $U\times V$ sia riducibile, il che significa che entrambe le sue proiezioni
    sono riducibili. Per ipotesi induttiva, $\pi^1 t$ \`e fortemente
    normalizzante. Similmente a quanto gi\`a fatto, notiamo che $\nu(t)\leq
    \nu(\pi^1 t)$ perch\'e ad ogni catena $t \conv t_1 \conv \cdots \conv t_k$
    possiamo associare la catena $\pi^1 t \conv \pi^1 t_1\conv \cdots \conv
    \pi^1 t_k$. Ne consegue che $\nu(t)$ \`e finito e questo dimostra (CR1).

    Per (CR2) osserviamo che se $t$ e $t'$ sono termini per cui $t \conv t'$,
    allora $\pi^1 t \conv \pi^1 t'$ e allora $\pi^2 t \conv \pi^2 t'$. Se $t$
    \`e riducibile, allora lo sono anche le sue proiezioni e quindi, per ipotesi
    induttiva, lo sono anche le riduzioni delle proiezioni. Poich\`e $t'$ ha
    entrambe le proiezioni riducibili, \`e anch'esso riducibile.

    Infine sia $t$ un termine neutrale per cui tutte le sue conversioni sono
    riducibili.  Convertendo $\pi^1 t$, si pu\`o ottenere solo un termine della
    forma $\pi^1 t'$ con $t'$ conversione di $t$, in quanto $t$ \`e neutrale e
    non pu\`o essere una coppia. Allora $t'$ \`e riducibile e quindi anche
    $\pi^1 t'$ lo \`e. Ma questo vuol dire che tutte le conversioni del termine
    neutrale $\pi^1 t$ sono riducibili, che implica che anche $\pi^1 t$ \`e
    riducibile. Ripetendo il ragionamento con la seconda proiezione si dimostra
    la riducibilit\`a di $t$ e quindi anche (CR3).
\end{proof}

Notiamo che $O$ \`e un termine di tipo atomico in forma normale, e pertanto \`e
riducibile. Inoltre se $t$ \`e un termine di tipo $\tint$ riducibile (quindi
fortemente normalizzante) allora anche $St$ \`e riducibile. Infatti vale che
$\nu(St)=\nu(t)$.

Dimostriamo ora una coppia di lemmi simili al lemma \deb{Lemma??}.

\begin{block}[Lemma]
    Se $u$ e $v$ sono riducibili, anche $\ang{u,v}$ \`e riducibile.
\end{block}

\begin{proof}
    Ragioniamo per induzione sulla somma $\nu(u)+\nu(v)$ per dimostrare che
    $\pi^1\ang{u,v}$ si converte a soli termini riducibili, e quindi \`e
    anch'esso riducibile. Tale termine si pu\`o convertire in:
    \begin{itemize}
        \item $u$, che \`e riducibile.
        \item $\pi^1\ang{u',v}$ con $u'$ conversione di $u$. In questo caso
            abbiamo che $u'$ \`e riducibile per (CR2), e quindi poich\`e
            $\nu(u')<\nu(u)$ otteniamo che per ipotesi induttiva la
            riducbilit\`a di $\pi^1\ang{u',v}$.
        \item $\pi^1\ang{u,v'}$, con $v'$ conversione di $v$. Questo caso \`e
            identico al precedente.
    \end{itemize}
    Ripetendo la dimostazione con $\pi^2\ang{u,v}$, otteniamo che la coppia
    $\ang{u,v}$ \`e riducibile.
\end{proof}

\begin{block}[Lemma]
    Se $u$, $v$ e $t$ sono termini riducibili, allora, se i tipi dei
    sottotermini ne permettono la costruzione, anche $Ruvt$ \`e un termine
    riducibile.
\end{block}

\begin{proof}
    Di nuovo ragioniamo per induzione per ottenere che tutte le conversioni del
    termine $Ruvt$ sono riducibili per poi concludere con (CR3). In questo caso
    l'induzione viene fatta su $\nu(u) + \nu(v) + \nu(t) + l(t)$, dove $l(t)$
    indica il numero di simboli che appaiono nella forma normale di $t$.

    Il termine $Ruvt$ si converte in:
    \begin{itemize}
        \item $Ru'v't'$, con $u'$, $v'$ e $t'$ riduzioni rispettivamente di $u$,
            $v$ e $t$. In questo caso vale che $\nu(u') + \nu(v') + \nu(t')
            < \nu(u) + \nu(v) + \nu(t)$ e $l(t)=l(t')$. Si pu\`o quindi usare
            l'ipostesi induttiva per concludere che $R u'v't'$ \`e riducibile.
        \item $u$, riducibile per ipotesi. Questo accade quando $t=O$. 
        \item $v(Ruvw)w$, dove $t=Sw$. In questo caso avevamo gi\`a osservato
            che $\nu(t) = \nu(w)$, ma vale anche che $l(w) < l(w)$. L'ipotesi
            induttiva dice dunque che $Ruvw$ \`e riducibile. Poich\`e anche $v$
            e $w$ sono riducibili, per definizione di riducibilit\`a del tipo
            freccia anche $v(Ruvw)w$ \`e riducibile.
    \end{itemize}
    Concludiamo che $Ruvt$ \`e riducibile.
\end{proof}

Occorre a questo punto estendere la \deb{Proposizione??} con ulteriori casi per
gestire i termini introdotti nel sistema T.

\begin{proof}
    \begin{itemize}
        \item Se $t = \pi^1 w$, allora per ipotesi induttiva
            $w\sub{\overline{u}} {\overline{x}}$ \`e riducibile. Per cui anche 
            $\pi^1 (w\sub{\overline{u}} {\overline{x}}) = t \sub{\overline{u}}
            {\overline{x}}$ \`e riducibile.
        \item Se $t = \pi^1 w$, allora il caso \`e identico al precedente.
        \item Se $t = \ang{v,w}$, allora per ipotesi induttiva
            $v\sub{\overline{u}} {\overline{x}}$ e $w\sub{\overline{u}}
            {\overline{x}}$ sono riducibili. La tesi qui si ha come conseguenza
            del \deb{Lemma pi\`u su}.
        \item Se $t=O$, allora abbiamo gi\`a notato che \`e riducibile.
        \item Se $t = Sw$, abbiamo che se $w\sub{\overline{u}} {\overline{x}}$
            \`e riducibile, allora $t\sub{\overline{u}} {\overline{x}}$ \`e
            fortemente normalizzante e quindi riducibile.
        \item Se $t = Rvwp$, allora per ipotesi anche le sostituzioni nei tre
            paramentri sono riducibili, e quindi per il \deb{Lemma??}, anche il
            termine $t\sub{\overline{u}} {\overline{x}}$ \`e riducibile.
    \end{itemize}
\end{proof}

Come nel caso del calcolo tipato semplice, anche in questo caso abbiamo il
teorema come corollario dell'ultima proposizione.

\subsection{Rappresentabilit\`a nel Sistema T}

Prima di studiare risultati generali sulla rappresentabilit\`a di funzioni nel
sistema T di G\"odel, mostriamo qualche esempio di funzione rappresentabile.

La somma pu\`o essere rappresentata dal termine
\[
    A = \l p^\tint q\tint. Rp (\l x^\tint y^\tint. Sx) q.
\]
Infatti vediamo che soddisfa la definizione ricorsiva della somma:
\begin{gather*}
    ApO \conv R p (\l x^\tint y^\tint. Sx) O \conv p
\end{gather*}
e 
\begin{gather*}
    Ap(Sq) \conv R p (\l x^\tint y^\tint. Sy) (Sq) \conv \\ (\l x^\tint y\tint. Sy)(R p (\l x^\tint y^\tint. Sy) q) q \conv \\
    S (A pq)
\end{gather*}

\section{Il sistema F}

In questa sezione introduciamo un'ulteriore variante del $\l$-calcolo tipato
detta sistema F. Essa mantiene e anzi estende l'espressivit\`a del sistema T di
G\"odel e allo stesso tempo evita soluzioni ad hoc come l'inserimento fra i tipi
di un tipo $\tint$. Nel sistema F vedremo che esiste comunque un tipo
corrispondente ai numeri naturali, ma la sua costruzione \`e assai pi\`u
generale, e permette, fra le tante cose, di definire tipi coppia e tipi unione
oppure tipi corrispondenti a liste o alberi. La chiave per fare questo \`e
l'introduzione di una quantificazione sui tipi.

\begin{block}[Definizione]
    I tipi del sistema F sono definiti induttivamente come:
    \begin{itemize}
        \item $U_1, U_2, \dots$ sono tipi, detti variabili di tipo e sono anche
            detti tipi atomici.
        \item Se $U$ e $V$ sono tipi, allora anche $(U\rar V)$ \`e un tipo.
        \item Se $X$ \`e una variabile di tipo e $U$ \`e un tipo, allora anche
            $\P X. U$ \`e un tipo.
    \end{itemize}
\end{block}

Avendo una quantificazione \`e utile avere anche la distinzione tra occorrenze
libere e legate di variabili di tipo. Un'occorrenza di una variabile di tipo $X$
\`e legata se $X$ compare in un tipo nella forma $\P X. U$.

\begin{block}[Definizione]
    I termini del sistema F si definiscono induttivamente come:
    \begin{itemize}
        \item Per ogni tipo $U$, le variabili $x^U_1, x^U_2, \dots$ sono termini
            di tipo $U$.
        \item Se $t$ e $v$ sono termini di tipo rispettivamente $U\rar V$ e $U$,
            allora l'applicazione $(tv)$ \`e un termine di tipo $V$.
        \item Se $t$ \`e un termine di tipo $V$ e $x$ \`e una variabile di tipo
            $U$, allora l'astrazione $(\l x.  t)$ \`e un termine di tipo $U\rar
            V$.
        \item Se $u$ \`e un termine di tipo $U$ e $X$ \`e una variabile di tipo
            che non occorre libera nei tipi delle variabili libere di $u$,
            allora $\L X. u$ \`e un termine di tipo $\P X.U$. 
        \item Se $t$ \`e un termine di tipo $\P X. U$ e $V$ \`e un tipo allora
            $tV$ \`e un termine di tipo $U\sub{V}{X}$, che indica il termine $U$
            in cui sono state sostituite le occorrenze libere di $X$ con $V$,
            con l'usuale attenzione per la cattura di variabili.
    \end{itemize}
\end{block}

Occorre ancora definire il comportamento dei nuovi termini sotto l'azione della
conversione.
\begin{block}[Definizione]
    Un termine $u$ si converte a $v$ quando $v$ \`e ottenuto sostituendo in $u$
    un sottotermine $u'$ con un termine $v'$ tali che valga una delle seguenti:
    \begin{itemize}
        \item $u'= (\l x. w)t$ e $v' = w\sub{w}{x}$ dove $x$ \`e una variabile e
            $w$ e $t$ sono termini tali che la sostituzione $w\sub{w}{x}$ sia
            permessa.
        \item $u' = (\L X.v)U$ e $v'=v\sub{U}{X}$, con $v$ un termine, $U$ un
            tipo e $X$ una variabile di tipo.
    \end{itemize}
\end{block}

Attraverso l'astrazione sui tipi, o generalizzazione, possiamo comporre
programmi che operano uniformemente su dati di differenti tipi. L'esempio pi\`u
facile \`e dato dal termine identit\`a $\l x. \l x^x. x$, di tipo $\P X. X$.
Dato un termine $u$ di tipo $U$, si ha che $(\l x. \l x^x. x)U$ ha tipo $U$ e si
riduce a $\l x^U.x$ e quindi $(\l x. \l x^x. x)Uu$ si riduce semplicemente a
$u$.

Altri tipi che possono essere costruiti corrispondo al tipo prodotto e il tipo
unione disgiunta. Per esempio un tipo prodotto tra $U$ e $V$ pu\`o essere
rappresentato nel sistema F dal tipo $U\times V = \P X. (U\rar V \rar X) \rar
X$. Possiamo scrivere un termine che produce una coppia, ovvero
\[
    \text{CPL} = \l x^U  y^V. \L X. \l p^{U \rar V\rar X}. puv
\]
e le conseguenti proiezioni
\begin{gather*}
    \pi^1 = \l x^{U\times V}. xU(\l y^U z^V.y) \\
    \pi^2 = \l x^{U\times V}. xV(\l y^U z^V.z)
\end{gather*}
rispettivamente sul primo e secondo fattore.

Mostriamo per esempio la riduzione di 
\begin{gather*}
    \pi^1 (\text{CPL}uv) \conv \\
    (\l x^{U\times V}. xU(\l y^U z^V.y))(\text{CPL}uv) \conv \\
    (\text{CPL}uv)U (\l y^U z^V.y) \conv \\
    (\L X. \l p^{U\rar V \rar X} puv)U (\l y^U z^V.y)  \conv \\
    (\l y^U z^V.y)uv \conv u.
\end{gather*}

Chiaramente tutti i termini precedenti possono essere generalizzati su $U$ e su
$V$.

Il tipo somma tra $U$ e $V$ \`e definito come il tipo $U+V = \P X. (U\rar X)
\rar (V\rar X) \rar X$ con il costruttore con la variabile di tipo $U$ come 
\[
    \text{UN1} = \l u^U. \L X. \l p^{U\rar X} q^{V\rar X}. pu
\]
e l'equivalente sulla variabile di tipo $V$ come 
\[
    \text{UN2} = \l v^V. \L X. \l p^{U\rar X} q^{V\rar X}. qv.
\]
Mostriamo anche un termine di elimiazione per il tipo unione, il termine 
\[
    D uvt = tU (\l x^U.u)(\l x^V.v)
\]
tale che valga la riduzione
\begin{gather*}
    Duv(\text{UN1}r) = (\L X. \l p^{U\rar X}q^{V\rar X}. pr) U (\l x^U.u)(\l
    x^V.v) \conv \\
    (\l p^{U\rar X}q^{V\rar X}. pr)(\l x^U.u)(\l x^V.v) \conv \\
    (\l x^U.u) r \conv u\sub{r}{x}
\end{gather*}
e l'altra equivalente $Duv(\text{UN2}r) \conv v\sub{r}{x}$.
L'idea intuitiva del tipo somma corrisponde alle union del linguaggio C, e il
termine $D$ serve per distinguere in quale dei due tipi l'ultimo parametro \`e
utilizzato.

La costruzione di questi due tipi \`e in realt\`a generalizzabile a qualunque
tipo di dato algebrico. Ne mostriamo un ultimo esempio, pi\`u chiaro. Supponiamo
di voler costruire un tipo relativo agli alberi binari in cui ogni nodo contiene
un dato di tipo $U$. L'albero \`e generato utilizzando due costruttori. Il
primo, che chiamiamo $E$, \`e l'albero vuoto. Il secondo, $R$, ha come parametri
il dato da mettere nella radice dell'albero e i due alberi figli della radice.
Se chiamassimo $X$ il tipo dell'albero binario, risulterebbe che i costruttori
avrebbero tipo rispettivamente tipo $X$ e $U \rar X\rar X \rar X$. Da qui
costruiamo l'effettivo tipo dell'albero
\[
    T = \P X. X \rar (U \rar X \rar X \rar X) \rar X.
\]
I due costruttori sono rappresentati dai termini:
\begin{gather*}
    E = \L X. (\l x^X y^{U \rar X \rar X \rar X}.x) \\
    R = \l r^U l^T r^T \L X. (\l x^X y^{U \rar X \rar X \rar X}.yr(lXxy)(rXxy))
\end{gather*}

Se $u_1, u_2,\dots$ sono termini di tipo $U$ un esempio di albero binario \`e 
\[
    Ru_1(Ru_2(Ru_4)E)(Ru_3EE).
\]

\`E possibile operare su queste strutture con funzioni definite con
pattern-matching. Per esempio se abbiamo un termine $f$ di tipo $U\rar V\rar V$
e un termine iniziale $v$ di tipo $V$, possiamo definire l'iteratore di visita
anticipata sull'albero $t$ come:
\[
    Ivft = tVv(\l n^U l^V r^v. f(f(fl)n)r)
\]
\deb{spiegare un minimo}


Notiamo tuttavia che l'introduzione del sistema F porta con s\`e un difficolt\`a
nel lavorare per induzione sulla complessit\`a dei termini. Infatti nell'esempio
dell'identit\`a, il termine pu\`o essere istanziato su un qualunque tipo, e
dunque l'istanziazione non riduce la complessit\`a di un termine, ma
potenzialmente la pu\`o aumentare. Per esempio questo accade nel termine $(\L X.
\l x^X. x)(\P X. X)$. Nello studio del sistema F ci imbatteremo in problemi di
questo genere e presenteremo dei modi per aggirarli.

\subsection{Normalizzazione per il Sistema F}
Per dimostrare la normalizzazione forte nel sistema F, un primo tentativo pu\`o
essere quello di estendere la dimostrazione gi\`a fatta per il $\l$-calcolo
tipato semplice e per il sistema T. Questo per\`o non \`e possibile, perch\`e
nello spirito della definizione \deb{la definzione di riducibilit\`a}, vorremmo
provare a definiri i riducibili di tipo $\P X. V$ come i termini $t$ tali che
per ogni tipo $U$ il termine $tU$ \`e riducibile di tipo $V\sub{U}{X}$. Questo
conduce a una definizione impredicativa, perch\`e per conoscere la
riducibilit\`a di un termine di tipo $V\sub{U}{X}$ occorre conoscere la
riducibilit\`a dei suoi sottotermini. Nel caso in cui per esempio si avesse $U =
\P X.V$, le dimostrazioni per induzione fallirebbero. Occorre dunque un
procedimento adatto ad aggirare il problema.

Come prima cosa, estendiamo la definizione di riducibilit\`a del calcolo tipato
semplice con i nuovi casi introdotti nel sistema F.
\begin{block}[Definizione]
    Un termine $t$ si dice neutrale se \`e in una delle seguenti forme: $x$,
    $vu$ o $vU$, in cui $x$ \`e una variabile, $v$ e $u$ sono termini e $U$ \`e
    un tipo.
\end{block}

A questo punto possiamo definire i candidati di riducibilit\`a. Essi sono
insiemi di termini di uno stesso tipo per cui valgono le tre propriet\`a che
avevamo dimostrato valere per i riducibili. L'idea quindi \`e quella di
costruire degli ulteriori insieme di termini detti riducibili parametrici, la
cui costruzione induttiva sui tipi atomici e sui tipi freccia corrisponde a
quella data per i riducibili nel tipato semplice, e sui tipi della forma $\L
X.V$ corrisponde alla proposta fatta sopra, per\`o utilizzando come parametri,
al posto dei riducibili, dei generici candidat.

\begin{block}[Definizione]
    Un candidato di riducibilit\`a (o semplicemente candidato) di tipo $U$ \`e
    un insieme $\can{R}$ di termini di tipo $U$ per cui valgono:
    \begin{nlist}[CR1]
        \item Se $t\in\can{R}$ allora $t$ \`e fortemente normalizzante.
        \item Se $t\in\can{R}$ e $t'$ \`e un termine ottenuto da una riduzione
            di $t$, cio\`e $t\reduce t'$, allora $t'\in\can{R}$.
        \item  Se $t$ \`e neutrale, e per ogni conversione di uno step di $t$ si
            ottiene un termine $t'\in\can{R}$, allora anche $t\in\can{R}$.
    \end{nlist}
\end{block}

Definiamo una notazione semplice per i candidati di tipo freccia costruiti
esattamente come i riducibili del calcolo tipato semplice.
\begin{block}[Definizione]
    Se $\can{R}$ e $\can{S}$ sono insiemi di termini di tipo rispettivamente $U$
    e $V$, si definisce l'insieme $\can{R}\rightarrow\can{S}$ come l'insieme dei
    termini di tipo $U\rightarrow V$ per cui per ogni termine $u\in\can{R}$ si
    ha che $tu\in\can{S}$.
\end{block}

Occorre dimostrare che questi candidati siano effettivamente dei candidati, e lo
facciamo nel seguente lemma.
\begin{block} [Lemma]
    Se $\can{R}$ e $\can{S}$ sono candidati per i tipi $U$ e $V$, allora
    $\can{R}\rightarrow\can{S}$ \`e candidato di tipo $U\rightarrow V$.
\end{block}
\begin{proof}
    Per mostrare (CR1) prendiamo $t\in\can{R}\rightarrow\can{S}$ e una variabile
    $x$ di tipo $U$. Poich\'e le variabili sono sia normali che che neutrali,
    $x\in \can{R}$ e quindi $tx\in\can{S}$. Inoltre $\nu(t)<\nu(tx)$, e quindi
    siccome $tx$ \`e fortemente normalizzante, anche $t$ lo \`e.

    Per (CR2),se $t\reduce t'$, per ogni $u \in \can{R}$ si ha che $tu\reduce
    t'u$. Usando la (CR2) su $\can{S}$, si ottiene che $t'u\in\can{S}$. Allora
    $t'\can{R}\rightarrow\can{S}$.

    Infine consideriamo $t$ neutrale di tipo $U\rightarrow V$ per cui per tutte
    le conversioni di uno step $t\reduce t'$ si ha che $t' \in
    \can{R}\rightarrow\can{S}$. Sia $u\in\can{R}$, e per induzione su $\nu(u)$
    dimostriamo che $tu$ si riduce in uno step a termini in $\can{S}$. Infatti
    poich\'e $t$ \`e normale, $tu$ si pu\`o ridurre solo a $t'u$ o a $tu'$ per
    opportuni termini $t'$ e $u'$. Ma il primo appartiene a $\can{S}$ perch\`e
    $t'\in\can{R}\rightarrow\can{S}$, e il secondo ci appartiene per ipotesi
    induttiva in quanto $\nu(u')<\nu(u)$. Per (CR3) su $\can{S}$ allora
    $tu\in\can{S}$. 
\end{proof}

Possiamo dunque definire cosa sono i riducibili parametrici:
\newcommand{\subrx}{\sub{\underline{\can{R}}}{\underline{X}}}
\begin{block} [Definizione]
    Sia $T[\underline{X}]$ un tipo con variabili libere in $\underline{X}$. Sia
    $\underline{U}$ un vettore di tipi della stessa lunghezza e siano $\can{R}$
    dei rispettivi candidati. Possiamo allora definire l'insieme
    $\red{T}\subrx$ di termini riducibili parametrici di tipo
    $T\sub{\underline{U}}{\underline{X}}$ nel modo seguente:
    \begin{nlist} [1]
        \item  Se $T=X_i$ per qualche indice $i$, allora
            $\red{T}\subrx=\can{R}_i$.
        \item Se $T=V\rightarrow U$, allora $\red{T}\subrx = \red{V}\subrx
            \rightarrow \red{W}\subrx$.
        \item Se $T=\Pi Y.W$, allora $\red{T}\subrx$ \`e l'insieme dei termini
            $t$ di tipo $\sub{\underline{U}}{\underline{X}}$ tali che per ogni
            tipo $V$ e per ogni candidato $\can{S}$ di tale tipo vale che $tV
            \in \red{W}\subrx\sub{\can{S}}{Y}$.
    \end{nlist}
\end{block}

Ci servir\`a nel seguito utilizzare induttivamente i riducibili parametrici come
parametri di altri riducibili paramentrici. Nel prossimo lemma dimostreremo che
\`e possibile farlo in quanto i riducibili parametrici rispettano le propriet\`a
di candidati.
\begin{block} [Lemma]
    $\red{T}\subrx$ \`e un candidato di riducibilit\`a di tipo
    $T\sub{\underline{U}}{\underline{X}}$.
\end{block}
\begin{proof}
    Lo facciamo per induzione sulla complessit\`a del tipo $T$. Il caso in cui
    $T$ \`e una variabile individuale, il teorema \`e una tautologia. Il caso in
    cui $T=V\rightarrow W$ lo abbiamo gi\`a fatto. Manca solo il caso in cui
    $T=\Pi Y.W$. 

    Verifichiamo (CR1). Sia $t\in \red{T}\subrx$, $V$ un tipo e $\can{S}$ un suo
    candidato. Allora $tV \in \red{W}\subrx\sub{\can{S}}{Y}$ per definizione.
    Usando l'ipotesi induttiva sul tipo $W$ si ha che $tV$ \`e fortemente
    normalizzante. Ma vale anche che $\nu(t)<\nu(tV)$. Quindi anche $t$ \`e
    fortemente normalizzante. 

    Per (CR2), supponiamo di avere $t\reduce t'$ con uno step di conversione.
    Allora, $tV\reduce t'V$, per cui $t'V \in \red{W}\subrx\sub{\can{S}}{Y}$ e
    quindi $t'\in \red{T}\subrx$.

    Infine, per (CR3), consideriamo $t$ un qualunque termine di tipo $T$
    neutrale. Supponiamo che per ogni $t'$ ottenuto dalla conversione di $t$ in
    un singolo step si abbia $t'$ riducibile parametrico. Allora per ogni tipo
    $V$ e relativo condidato $\can{S}$, le uniche conversioni di $tV$ sono della
    forma $tV\reduce t'V$. Usando l'ipotesi induttiva allora anche $tV$ \`e
    riducibile parametrico, e quindi si ha la tesi.
\end{proof}

Il prossimo lemma \`e utile per studiare il comportamento delle riducibilit\`a
parametrica rispetto alle instaziazioni. Una cosa da notare \`e che affinch\`e
abbia senso il predicato $\red{T}\subrx \sub{\red{V}\subrx}{Y}$, occorre che
$\red{V}\subrx$ non sia soltanto un predicato, ma anche un effettivo insieme.
Notiamo anche che il principio di estensione ci permette di passare da predicati
a insiemi senza problemi.
\begin{block} [Lemmma]
    Sia $T$ un tipo con variabili libere $Y$ e $\underline{X}$ e $V$ un tipo.
    Siano $\underline{\can{R}}$ candidati per $\underline{X}$. Allora vale che
    $\red{T\sub{V}{Y}}\subrx= \red{T}\subrx \sub{\red{V}\subrx}{Y}$.
\end{block}
\begin{proof}
    Come prima, facciamo un'induzione sulla complessit\`a del tipo $T$. Per
    comodit\`a, usiamo l'abbreviazione $A=\red{V}\subrx$.

    Iniziamo con il caso in cui $T=Z$ \`e una variabile individuale diversa da
    $Y$. Allora vale che 
    \[
        \red{T\sub{V}{Y}}\subrx = \red{Z}\subrx = \red{Z}\subrx\sub{A}{Y}.
    \]

    Se invece $T=Y$ si ha che
    \[
        \red{T\sub{V}{Y}}\subrx = \red{V}\subrx = \red{Y}\subrx\sub{
            \sub{V}\subrx}{Y}.
    \]

    Consideriamo ora il caso in cui $T=U\rightarrow W$. Vale che
    \begin{align*}
        \red{T\sub{V}{Y}}\subrx = 
        &\red{U\sub{V}{Y}\rightarrow W\sub{V}{Y}}\subrx =  \\
        &\red{U\sub{V}{Y}}\subrx \rightarrow \red{W\sub{V}{Y}}\subrx= \\
        &\red{U}\subrx\sub{A}{Y} \rightarrow\red{W}\subrx\sub{A}{Y} = \\
        &\red{U\rightarrow W}\subrx\sub{A}{Y}.
    \end{align*}

    Sia $Z$ come prima e svolgiamo il caso $T=\Pi Z. W$. Per definizione,
    $\red{\Pi Z. W\sub{V}{Y}}\subrx$ \`e l'insieme di tutti i termini $t$ per
    cui per ogni tipo $U$ e relativo candidato $\can{S}$ vale che
    \[
        tU\in \red{W\sub{V}{Y}}\subrx\sub{\can{S}}{Z} = 
        \red{W}\subrx\sub{\can{S}}{Z}\sub{A}{Y}.
    \]
    Dunque si ottiene la tesi per la definizione di $\red{\Pi Z.W}$.

    Infine il caso in cui $T=\Pi Y.W$ \`e semplice perch\`e $Y$ non occorre
    libera in $T$.
\end{proof}

I seguenti due teoremi sono gli equivalenti per questa dimostrazione dei lemmi
\deb{quei lemmi}.
\begin{block} [Lemma]
    Se per ogni tipo $V$ e per ogni candidato di riducibilit\`a $\can{S}$ per
    $V$ vale che $w\sub{V}{Y}\in\red{W}\subrx\sub{\can{S}}{Y}$, allora $\Lambda
    Y. w\in\red{\Pi Y.W}\subrx$.
\begin{proof}
    Dimostriamo per induzione su $\nu(w)$ che tutte le conversioni in uno step
    di $(\Lambda Y.w)V$ sono in $\red{W} \subrx\sub{\can{S}}{Y}$. Una
    conversione di tali conversioni possono essere soltanto di due forme.  La
    prima \`e $(\Lambda Y.w')V$, con $w'$ una conversione di $w$. Ma allora
    $\nu(w')<\nu(w)$ e si usa l'ipotesi induttiva. La seconda forma \`e del tipo
    $w\sub{V}{Y}$, e questa \`e riducibile parametrico per ipotesi del lemma. 

    Allora la dimstrazione si conclude per (CR3).
\end{proof}
\end{block}

\begin{block} [Lemma]
    Se $t\in \red{\Pi Y.W}\subrx$, allora $tV\in \red{W\sub{V}{Y}}\subrx$ per
    ogni tipo $V$.
\begin{proof}
    Per la definizione di $\red{\Pi Y.W}$, per ogni candidato $\can{S}$ per $V$
    vale che $tV\in \red{W}\subrx\sub{\can{S}}{Y}$. Allora vale anche per
    $\can{S} = \red{V}\subrx$ e la tesi segue per il \deb{lemma XXX}.
\end{proof}
\end{block}

\begin{block} [Lemma]
    Se per ogni $u\in \red{U}\subrx$ vale che $v\sub{u}{x} \in \red{V}\subrx$,
    allora $\lambda x^U.v\in \red{U\rightarrow V}\subrx$.
\begin{proof}
    Dimostriamo per induzione su $\nu(u)+\nu(v)$ che tutte le conversioni di
    $(\lambda x^U.v)u$ sono riducibili parametrici. Infatti tale termine si
    converte in $(\lambda x^U.v)u'$, con $u'$ conversione di $u$, oppure in
    $(\lambda x^U.v')u$ con $v'$ conversione di $v$, oppure in $v\sub{u}{x}$. I
    primi due casi si risolvono con l'ipotesi induttiva, il terzo con l'ipotesi
    del lemma.

    Infine il teorema si dimostra per la propriet\`a (CR3).
\end{proof}
\end{block}

Utilizziamo ora l'idea di riducibilit\`a parametrica per definire i riducibili,
nello spirito che a posteriori potremo dire che i riducibili sono esattamente i
termini fortemente normalizzanti. In tal caso la definizione corrisponderebbe
con quella data per il $\l$-calcolo tipato semplice.
\begin{block} [Definizione]
    Un termine $t$ di tipo $T$ \`e riducibile se \`e in
    $\red{T}\sub{\underline{\can{SN}}}{\underline{X}}$ dove $X_1,\dots,X_m$ sono
    le variabili libere di $T$ e $\underline{\can{SN}}_i$ \`e l'insieme dei
    termini fortemente normalizzanti di tipo $X_i$.
\end{block}

Infine la proposizione che segue svolge lo stesso ruolo \deb{della proposizione
sul tipato semplice}.
\begin{block} [Proposizione]
    Sia $t$ un termine di tipo $T$ le cui variabili libere sono $x_1, \dots,
    x_n$ di tipo rispettivamente $U_1, \dots, U_n$. Supponiamo che le variabili
    libere dei tipi $T$ e di tutti gli $U_i$ siano $X_1, \dots, X_m$. Siano
    $\can{R}_1, \dots, \can{R}_m$ candidati di riducibilit\`a per dei tipi $V_1,
    \dots, V_m$ e siano inoltre $u_1, \dots, u_n$ termini di tipo
    $U_1\sub{\underline{V}}{\underline{X}}, \dots
    U_n\sub{\underline{V}}{\underline{X}}$ presi nei rispettivi
    $\red{U_i}\subrx$. Allora $t\sub{\underline{V}}{\underline{X}}
    \sub{\underline{u}}{\underline{x}} \in \red{T}\subrx$.
\newcommand{\subvx}{\sub{\underline{V}}{\underline{X}}}
\newcommand{\subut}{\sub{\underline{u}}{\underline{x}}}
\begin{proof}
    Per induzione sulla complessit\`a di $t$. Distinguiamo allora i seguenti
    casi:
    \begin{nlist}[i]
        \item $t=x_i$. Questo caso \`e una tautologia.
        \item $t=wv$, con $w$ di tipo $W\rightarrow T$ e $v$ di tipo $W$. Per
            ipostesi induttiva vale che $w\subvx\subut\in \red{W \rightarrow
            T}\subrx$ e che $v\subvx\subut \in \red{W}\subrx$. In questo caso la
            tesi segue dalla definizione di $\red{W\rightarrow T}$.
        \item $t=wS$. Questo caso \`e una diretta conseguenza del lemma XXX
            sull'istanziazione. 
            % @todo
        \item $t=\Lambda Z. Y$. Questo discende dal lemma XXX sulla
            generalizzazione.
        \item $t=\lambda y^P.w$. Questo caso si fa con il lemma XXX sui tipi
            freccia.
    \end{nlist}
\end{proof}
\end{block}

Come corollari otteniamo il seguente risultato e il teorema di normalizzazione
forte per il sistema F.

\begin{block} [Proposizione]
    Tutti i termini del sistema F sono riducibili.
\begin{proof}
    Basta usare la proposizione precedente e prendere $\can{R}_i=\can{SN}_i$ e
    $u_i=x_i$.
\end{proof}
\end{block}

\begin{block} [Teorema]
    Tutti i termini del sistema F sono fortemente normalizzanti.
\end{block}


\section{Aritmetiche di Peano e di Heyting}

\begin{block}[Definizione]
    Il linguaggio per la logica del secondo ordine \`e lo stesso di quello del
    primo ordine con l'aggiunta per ogni naturale $n$ di numerabili simboli
    $X^n$, che chiameremo variabili di relazione. 
    
    Le formule atomiche sono $\bot$ e espressioni della forma $X^n(t_1, \dots,
    t_n)$, dove i $t_i$ sono termini del linguaggio.

    Le formule sono definite induttivamente come:
    \begin{itemize}
        \item Le formule atomiche.
        \item Date $\phi$ e $\psi$ formule, sono formule anche $\phi \land
            \psi$, $\phi \lor \psi$, $\phi \rar \psi$.
        \item Data una formula $\phi$ e una variabile $x$, sono formule anche
            $\forall x \phi$ e $\exists x \phi$.
        \item Data una formula $\phi$ e una variabile di relazione $X$, sono
            formule anche $\forall X \phi$ e $\exists X \phi$.
    \end{itemize}
    Si definisce inoltre la formula $\lnot \phi$ come $\phi \rar \bot$.
\end{block} 

In modo naturale possiamo definire il concetto di variabili libere in una
formula:
\begin{block}[Definizione]
    \begin{itemize} 
        \item Le variabili libere di $X(t_1, \dots, t_n)$ con $X$ variabile di
            relazione $n$-aria, sono l'unione di tutte le variabili libere che
            compaiono nei termini $t_i$ per ogni $i$ e $X$.
        \item Le variabili libere di $r(t_1, \dots, t_n)$ con $r$ simbolo d
            relazione $n$-aria sono l'unione di tutte le variabili libere che
            compaiono nei termini $t_i$ per ogni $i$.
        \item Le variabili libere di $\phi \land \psi$, $\phi \lor \psi$, $\phi
            \rar \psi$ sono l'unione delle variabili libere di $\phi$ e $\psi$.
        \item Le variabili libere di $\forall x \phi$ e $\exists x \phi$ con $x$
            variabile, sono le variabili libere di $\phi$ meno $x$.
        \item Le variabili libere di $\forall X \phi$ e $\exists X \phi$ con $X$
            variabile di relazione, sono le variabili libere di $\phi$ meno $X$.
    \end{itemize} 
\end{block}

La sostituzione di termini nelle variabili \`e la sostituzione standard, con
l'attenzione di evitare la cattura delle variabili. Per sostituire invece
relazioni al posto di variabili di relazione ci appoggeremo al concetto di
specie.

\begin{block}[Definizione]
    Sia $\phi$ una formula e $x_1, \dots, x_n$ delle variabili individuali,
    allora l'espressione $\l x_1, \dots, x_n.\phi$ \`e una specie di ariet\`a
    $n$. Si noti che le variabili $x_i$ possono apparire o non apparire in
    $\phi$. 

    Le variabili libere di $\l x_1, \dots, x_n.\phi$ sono le variabili libere di
    $\phi$ meno le variabili $x_i$.

    Abbrevieremo inoltre l'espressione $\l x_1, \dots x_n.X(x_1, \dots, x_n)$
    con semplicemente $X$ e se $\underline{x}=x_1,\dots, x_n$, abbrevieremo $\l
    x_1, \dots, x_n.\phi$ con $\l\underline{x}.\phi$.

    Definiamo inoltre induttivamente la sostituzione di una specie di ariet\`a
    $n$ $\l\underline{x}.\phi$ in una variabile di relazione $n$-aria $X$:
    \begin{itemize}
        \item $\bot\sub{\l\underline{x}.\phi}{X} = \bot$.
        \item $r(t_1, \dots, t_n)\sub{\l\underline{x}.\phi}{X} = r(t_1, \dots,
            t_n)$ quando $r$ \`e una relazione oppure una variabile di relazione
            diversa da $X$.
        \item $(X(\underline{t})\sub{\l\underline{x}.\phi}{X} =
            \phi\sub{\underline{t}}{\underline{x}}$.
        \item $(\eta\rar\psi)\sub{\l\underline{x}.\phi}{X} =
            \eta\sub{\l\underline{x}.\phi}{X} \rar
            \psi\sub{\l\underline{x}.\phi}{X}$
            e equivalentemente per $\eta\lor\psi$ e $\eta\land\psi$.
        \item $(\forall x \eta)\sub{\l\underline{x}.\phi}{X} = \forall x
            \eta\sub{\l\underline{x}.\phi}{X}$ per tutte le variabili
            individuali $x$ che non appaiono libere in $\l\underline{x}.\phi$.
            Equvalentemente per $\exists x \eta$.
        \item $(\forall Y \eta)\sub{\l\underline{x}.\phi}{X} = \forall Y
            \eta\sub{\l\underline{x}.\phi}{X}$ con $Y$ variabile di relazione
            diversa da $X$ e $Y$ che non appare libera in
            $\l\underline{x}.\phi$.
    \end{itemize}
\end{block}

A questo punto presentiamo le regole della deduzione naturale per la
logica del secondo ordine:
\deb{regole per la deduzione, p.308}

Aggiungendo queste regole alle regole della deduzione naturale per la logica
classica del primo ordine, si ottiene il sistema per la logica classica del
secondo ordine. Equivalentemente, aggiungendole alle regole per la logica
intuizionista del primo ordine si ottiene il sistema per la logica intuizionista
del secondo ordine.

Notiamo che \`e possibile dimostrare in entrambi i tipi di logica il principio
di comprensione:
\[
    \exists Y\ \forall x\ (\phi(x) \leftrightarrow x\in Y).
\]
per ogni formula $\phi$.

Inoltre molti dei connettivi presentati sono ridondanti: infatti \`e possibile
definirli tutti in termini dei soli $\rar$ e $\forall$ (su variabili individuali
e di relazione). In particolare: \deb{\`E corretto usare =?}
\begin{itemize}
    \item $\bot = \forall X.X$.
    \item $\phi \lor \psi = \forall X ((\phi\rar X)\rar (\psi\rar X) \rar X)$.
    \item $\phi \land \psi = \forall X ((\phi\rar\psi\rar X) \rar X)$.
    \item $\exists x \phi = \forall R (\forall x(\phi\rar R)\rar R)$.
    \item $\exists X \phi = \forall R (\forall X(\phi\rar R)\rar R)$.
\end{itemize}

Il prossimo passo \`e quello di mettere in evidenza un rapporto che sussiste tra
le proposizioni derivabili dalla logica intuizionista e la logica classica.
\begin{block}[Definizione]
    Data una formula $\phi$ definiamo induttivamente la sua traduzione di
    G\"odel $k(\phi)$ come:
    \begin{itemize}
        \item $\lnot\lnot \phi$ se $\phi$ \`e atomica.
        \item $\lnot\lnot(k(\eta)\rar k(\psi))$ se $\phi=\eta\rar\psi$, e in
            modo di equivalentemente si definisce per gli altri connettivi
            binari.
        \item \dots
    \end{itemize}
\end{block}

Introduciamo adesso le aritmetiche del secondo ordine. Utilizziamo un linguaggio
che ha come unica costante $0$, il simbolo di funzione successore $S$ e una
relazione binaria di uguaglianza $=$.

A questo punto possiamo dare il seguente risultato:
\begin{block}[Proposizione]
    La proposizione $\phi$ \`e un teorema della logica classica se e solo se
    $k(\phi)$ \`e un teorema della logica intuizionista.
\end{block}
\deb{Dimostrazione?}

Consideriamo i seguenti assiomi per l'uguaglianza:
\begin{nlist}[U1]
    \item $\forall a (a=a)$;
    \item $\forall ab (a=b\rar b=a)$;
    \item $\forall abc (a=b \rar b=c \rar a=c)$;
    \item $\forall ab(a=b \rar Sa = Sb)$,
\end{nlist}
in cui i primi tre sono i consueti assiomi per una relazione di equivalenza e
l'ultimo \`e una sorta di passo induttivo.

Aggiungiamo ancora tre assiomi di Peano:
\begin{nlist}[P1]
    \item $\forall ab(Sa =Sb \rar a = b)$;
    \item $\forall a(Sa=0 \rar \bot)$;
    \item $\forall a\ \text{Int}(a)$,
\end{nlist}
dove $\text{Int}(a) = \forall X(\forall b (X(b)\rar X(Sb)) \rar X(0) \rar X(a))$
serve a sostituire lo schema di induzione.

\begin{block}[Definizione]
    Gli assiomi (U1-4) e (P1-3) utilizzati con il sistema della logica classica
    del secondo ordine definiscono l'aritmetica di peano del secondo ordine PA2.
    Quando invece sono utilizzati con il sistema della logica intuizionista del
    secondo ordine definiscono l'aritmetica di Heyting del secondo ordine HA2.
\end{block}

Utilizzando la quantificazione al secondo ordine \`e anche possibili definire
dei predicati per la somma e il prodotto (e anche per le funzioni primitive
ricorsive).

Consideriamo $T$ una teoria nel linguaggio dell'aritmetica. Sia inoltre $\uni$
la funzione universale, ovvero una formula primitiva ricorsiva tale che $\uni
(e, n, m)$ \`e vera se e solo se il programma con codifica $e$ eseguito con
input $n$ ha output $m$. 

Diciamo che una funzione \`e dimostrabilmente totale in $T$ se esiste un
programma con codifica $e$ tale che
\[
    T \vdash \forall n\ \exists ! m\ \uni(e, n, m).
\]
\deb{Aggiungere precisazioni per funzioni con pi\`u di una variabile?}

Notiamo che la formula da dimostrare ha complessit\`a $\Pi^0_2$.

Vale il seguente teorema:
\begin{block}[Teorema]
    Le funzioni dimostrabilmente totali in PA2 sono esattamente le funzioni
    dimostrabilmente totali in HA2.
\end{block}
\deb{Dimostrazione?}


\section{Rappresentabilit\`a in F}

Nel sistema F \`e presente un tipo corrispondente ai numeri naturali, ovvero il
tipo
\[
    \tint = \Pi X. X \rar (X \rar X) \rar X
\]

dove si hanno i termini corrispondenti allo zero e al successore rispettivamente
uguali a
\[
    O = \L X. \l x. \l f. x
\]
\[
    S = \l n. \L X. \l x. \l f. f (nXxf).
\]

Possiamo scrivere allora i numerali come le forme normali di $S^nO$ per ogni $n$
naturale. A questo punto dimostriamo il lemma:

\begin{block}[Lemma]
    I numerali sono tutti e soli i termini in forma normale di tipo $\tint$. 
\end{block}

\deb{In realt\`a la costruzione fatta per i numeri naturali a partire dai
costruttori zero e successore pu\`o essere generalizzata a qualunque tipo di
dato algebrico.}

In modo equivalente a quanto gi\`a fatto con le altre versioni del $\l$-calcolo
\`e possibile definire la nozione di funzione rappresentabile, e poi dare una
caratterizzazione di tali funzioni.

\begin{block}[Teorema]
    Le funzioni dimostrabilmente totali in PA2 sono tutte e sole le
    funzioni rappresentabili nel sistema F.
\end{block}

Iniziamo con la freccia pi\`u semplice, ovvero $\Leftarrow$. Come nel caso del
sistema T, la dimostrazione della forte normalizzazione di un termine, pu\`o
essere \deb{interpretata} in PA2 come una dimostrazione della totalit\`a della
funzione corrispondente a tale termine. Infatti per la dimostrazione sono stati
utilizzati due principi:
\begin{itemize}
    \item Lo schema di comprensione, necessario a dimostrare che i riducibili
        parametrici sono candidati di riducibilit\`a. 
    \item Il principio di induzione.
\end{itemize}
Notiamo che tuttavia non \`e possibile esprimere la riducibilit\`a in generale,
ma solo per specifici termini.

\begin{block}[Lemma]
    Esiste un'unica deduzione normale di $\text{Int} (S^nO)$, ovvero
    $\check{n}$.
\end{block}

Consideriamo ora la formula $\phi(x,y)$ che esprime il fatto che dato un
algoritmo con input $x$ termini con output $y=f(x)$, a meno di una codifica con
numeral. Supponiamo di voler dimostrare in HA2
\[
    \forall n\in\N\ \exists m\in\N . \phi(x,y)
\]
ovvero
\[
    \forall x (\text{Int}(x)\rar \exists y. (\text{Int}(y)\land \phi(x,y))).
\]
Chiamiamo $\delta$ tale dimostrazione. AD essa associamo un termine
$\cod{\delta}$ di tipo $\tint\rar (\tint\times\cod{\phi})$ e un termine $t = \l
x^\tint.\pi^1(\cod{\delta}x)$ che ne contiene il significato algoritmico.


\end{document}
