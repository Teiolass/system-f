\documentclass[]{marticle}
\usepackage{mstyle}

\title{\textbf{\huge Rappresentabilit\`a di Funzioni nel Lambda-calcolo
Polimorfico}}
\date{}
\author{}


\begin{document}
\maketitle

\textbf{Candidato:} Alessio Marchetti

\textbf{Relatori:} Alessandro Berarducci, Marcello Mamino

\vspace{0.4cm}

Il $\l$-calcolo \`e un sistema formale sviluppato negli anni '30 da Alonzo
Church. Lo scopo originario era quello di fondare la matematica, e non \`e stato
raggiunto in quanto il sistema si rivel\`o inconsistente, come dimostrato da
Kleene e Rosser nel 1936. Un sottoinsieme di tale sistema si \`e comunque
sviluppato per la sua capacit\`a di esprimere computazioni mediante astrazione
su variabili e sostituzione. Lo studio del $\l$-calcolo \`e dunque lo studio di
entit\`a dette $\l$-termini che svolgono allo stesso tempo il ruolo di programmi
e di dati su cui i programmi lavorano. Sui termini si considerer\`a una
relazione di ordine, detta riduzione, che rappresenta l'esecuzione dei
programmi. All'interno del $\l$-calcolo, con opportune codifiche, \`e possibile
rappresentare i numeri naturali e tutte le funzioni calcolabili. Poich\`e
l'insieme delle funzioni calcolabili totali non \`e ricorsivamente enumerabile
esistono dei termini la cui computazione non termina (diremo che non sono
normalizzanti).

Il $\l$-calcolo tipato \`e una variante del $\l$-calcolo in cui ad ogni termine
\`e associata un'entit\`a sintattica detta tipo. Esso ha origine nei lavori di
Haskell Curry (1934) e di Church (1940).  La riduzione in questo caso \`e
ridefinita aggiungendo vincoli sul come \`e possibile comporre i termini in base
al loro tipo, e in particolare rende chiara la classe di dati su cui ciascun
programma pu\`o operare.  Come conseguenza si pu\`o dimostrare che questa
variante ha la propriet\`a di normalizzazione, ovvero tutte le computazioni
terminano e tutti i termini sono normalizzanti. I tipi sono studiati anche in
ambito informatico per la verifica in modo automatico della presenza di alcuni
errori che dovrebbero essere altrimenti cercati a mano dal programmatore.

In questa tesi ci occuperemo di una variante del calcolo detta sistema F, anche
nota come $\l$-calcolo polimorfico o $\l$-calcolo del secondo ordine. Essa \`e
stata sviluppata dal logico Jean-Yves Girard (1972) e dall'informatico John
Charles Reynolds (1974). Il sistema F \`e essenzialmente una variante del
$\l$-calcolo tipato in cui viene aggiunta una quantificazione universale sui
tipi.

Anche per il sistema F vale la propriet\`a di normalizzazione. Troveremo dunque
che le funzioni rappresentabili nel sistema F sono solo un sottoinsieme delle
funzioni calcolabili totali, e ne daremo una caratterizzazione pi\`u precisa:
esse sono esattamente le funzioni di cui l'aritmetica di Heyting del secondo
ordine dimostra la totalit\`a.  Mostreremo quindi un esempio di funzione non
rappresentabile nel sistema F e dedurremo dunque la consistenza dell'aritmetica
di Peano del secondo ordine a partire dal risultato di normalizzazione.

Metteremo inoltre tale risultato a confronto con un risultato equivalente su
un'altra variante del $\l$-calcolo, detta sistema T di G\"odel, per cui vale
ugualmente la propriet\`a di normalizzazione. Nel sistema T infatti le funzioni
rappresentabili sono esattamente quelle che l'aritmetica di Heyting del primo
ordine dimostra essere totali.

\end{document}

