\documentclass[]{marticle}
\usepackage{mstyle}

\title{\textbf{\huge Normalizzazione Forte per il Sistema F}}
\date{}


\begin{document}
\maketitle

\begin{block}[Definizione]
    Un termine $t$ si dice neutrale se \`e in una delle seguenti forme: $x$,
    $t'U$ o $t'U$, in cui $x$ \`e una variabile, $t'$ \`e un termine e $U$ \`e
    un tipo.
\end{block}

\begin{block}[Definizione]
    Un candidato di riducibilit\`a (o semplicemente candidato) di tipo $U$ \`e
    un insieme $\can{R}$ di termini di tipo $U$ per cui valgono:
    \begin{nlist}[CR1]
        \item Se $t\in\can{R}$ allora $t$ \`e fortemente normalizzabile.
        \item Se $t\in\can{R}$ e $t'$ \`e un termine ottenuto da una riduzione
            di $t$, cio\`e $t\reduce t'$, allora $t'\in\can{R}$.
        \item  Se $t$ \`e neutrale, e per ogni conversione di uno step di $t$ si
            ottiene un termine $t'\in\can{R}$, allora anche $t\in\can{R}$.
    \end{nlist}
\end{block}

\begin{block}[Definizione]
    Dato un termine $t$, si definisce $\nu(t)$ come la massimo numero di step
    di conversione necessari a portare $t$ in forma normale. In particolare
    $\nu(t)=\infty$ se e solo se $t$ non \`e fortemente riducibile.
\end{block}

\begin{block}[Definizione]
    Se $\can{R}$ e $\can{S}$ sono insiemi di termini di tipo rispettivamente $U$
    e $V$, si definisce l'insieme $\can{R}\rightarrow\can{S}$ come l'insieme dei
    termini di tipo $U\rightarrow V$ per cui per ogni termine $u\in\can{R}$ si
    ha che $tu\in\can{S}$.
\end{block}

\begin{block} [Lemma]
    Se $\can{R}$ e $\can{S}$ sono candidati per i tipi $U$ e $V$, allora
    $\can{R}\rightarrow\can{S}$ \`e candidato di tipo $U\rightarrow V$.
\end{block}
\begin{proof}
    Per mostrare (CR1) prendiamo $t\in\can{R}\rightarrow\can{S}$ e una variabile
    $x$ di tipo $U$. Poich\'e le variabili sono sia normali che che neutrali,
    $x\in \can{R}$ e quindi $tx\in\can{S}$. Inoltre $\nu(t)<\nu(tx)$, e quindi
    siccome $tx$ \`e fortemente normalizzabile, anche $t$ lo \`e.

    Per (CR2),se $t\reduce t'$, per ogni $u \in \can{R}$ si ha che $tu\reduce
    t'u$. Usando la (CR2) su $\can{S}$, si ottiene che $t'u\in\can{S}$. Allora
    $t'\can{R}\rightarrow\can{S}$.

    Infine consideriamo $t$ neutrale di tipo $U\rightarrow V$ per cui per tutte
    le conversioni di uno step $t\reduce t'$ si ha che $t' \in
    \can{R}\rightarrow\can{S}$. Sia $u\in\can{R}$, e per induzione su $\nu(u)$
    dimostriamo che $tu$ si riduce in uno step a termini in $\can{S}$. Infatti
    poich\'e $t$ \`e normale, $tu$ si pu\`o ridurre solo a $t'u$ o a $tu'$ per
    opportuni termini $t'$ e $u'$. Ma il primo appartiene a $\can{S}$ perch\`e
    $t'\in\can{R}\rightarrow\can{S}$, e il secondo ci appartiene per ipotesi
    induttiva in quanto $\nu(u')<\nu(u)$. Per (CR3) su $\can{S}$ allora
    $tu\in\can{S}$. 
\end{proof}

\end{document}
