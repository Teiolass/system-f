\documentclass[]{marticle}
\usepackage{mstyle}

\title{\textbf{\huge Normalizzazione Forte per il Sistema F}}
\date{}


\begin{document}
\maketitle

\begin{block}[Definizione]
    Un termine $t$ si dice neutrale se \`e in una delle seguenti forme: $x$,
    $t'U$ o $t'U$, in cui $x$ \`e una variabile, $t'$ \`e un termine e $U$ \`e
    un tipo.
\end{block}

\begin{block}[Definizione]
    Un candidato di riducibilit\`a (o semplicemente candidato) di tipo $U$ \`e
    un insieme $\can{R}$ di termini di tipo $U$ per cui valgono:
    \begin{nlist}[CR1]
        \item Se $t\in\can{R}$ allora $t$ \`e fortemente normalizzabile.
        \item Se $t\in\can{R}$ e $t'$ \`e un termine ottenuto da una riduzione
            di $t$, cio\`e $t\reduce t'$, allora $t'\in\can{R}$.
        \item  Se $t$ \`e neutrale, e per ogni conversione di uno step di $t$ si
            ottiene un termine $t'\in\can{R}$, allora anche $t\in\can{R}$.
    \end{nlist}
\end{block}

\begin{block}[Definizione]
    Se $\can{R}$ e $\can{S}$ sono candidati dei rispettivi tipi $U$ e $V$, si
    definisce l'insieme $\can{R}\rightarrow\can{S}$ come l'insieme dei termini
    di tipo $U\rightarrow V$ per cui per ogni termine $u\in\can{R}$ si ha che
    $tu\in\can{S}$.
\end{block}

\end{document}
