\documentclass[]{marticle}
\usepackage{mstyle}

\title{\textbf{\huge Normalizzazione Forte per il Sistema F}}
\date{}


\begin{document}
\maketitle

\begin{block}[Definizione]
    Un termine $t$ si dice neutrale se \`e in una delle seguenti forme: $x$,
    $t'U$ o $t'U$, in cui $x$ \`e una variabile, $t'$ \`e un termine e $U$ \`e
    un tipo.
\end{block}

\begin{block}[Definizione]
    Un candidato di riducibilit\`a (o semplicemente candidato) di tipo $U$ \`e
    un insieme $\can{R}$ di termini di tipo $U$ per cui valgono:
    \begin{nlist}[CR1]
        \item Se $t\in\can{R}$ allora $t$ \`e fortemente normalizzabile.
        \item Se $t\in\can{R}$ e $t'$ \`e un termine ottenuto da una riduzione
            di $t$, cio\`e $t\reduce t'$, allora $t'\in\can{R}$.
        \item  Se $t$ \`e neutrale, e per ogni conversione di uno step di $t$ si
            ottiene un termine $t'\in\can{R}$, allora anche $t\in\can{R}$.
    \end{nlist}
\end{block}

\begin{block}[Definizione]
    Dato un termine $t$, si definisce $\nu(t)$ come la massimo numero di step
    di conversione necessari a portare $t$ in forma normale. In particolare
    $\nu(t)=\infty$ se e solo se $t$ non \`e fortemente normalizzabile.
\end{block}

\begin{block}[Definizione]
    Se $\can{R}$ e $\can{S}$ sono insiemi di termini di tipo rispettivamente $U$
    e $V$, si definisce l'insieme $\can{R}\rightarrow\can{S}$ come l'insieme dei
    termini di tipo $U\rightarrow V$ per cui per ogni termine $u\in\can{R}$ si
    ha che $tu\in\can{S}$.
\end{block}

\begin{block} [Lemma]
    Se $\can{R}$ e $\can{S}$ sono candidati per i tipi $U$ e $V$, allora
    $\can{R}\rightarrow\can{S}$ \`e candidato di tipo $U\rightarrow V$.
\begin{proof}
    Per mostrare (CR1) prendiamo $t\in\can{R}\rightarrow\can{S}$ e una variabile
    $x$ di tipo $U$. Poich\'e le variabili sono sia normali che che neutrali,
    $x\in \can{R}$ e quindi $tx\in\can{S}$. Inoltre $\nu(t)<\nu(tx)$, e quindi
    siccome $tx$ \`e fortemente normalizzabile, anche $t$ lo \`e.

    Per (CR2),se $t\reduce t'$, per ogni $u \in \can{R}$ si ha che $tu\reduce
    t'u$. Usando la (CR2) su $\can{S}$, si ottiene che $t'u\in\can{S}$. Allora
    $t'\can{R}\rightarrow\can{S}$.

    Infine consideriamo $t$ neutrale di tipo $U\rightarrow V$ per cui per tutte
    le conversioni di uno step $t\reduce t'$ si ha che $t' \in
    \can{R}\rightarrow\can{S}$. Sia $u\in\can{R}$, e per induzione su $\nu(u)$
    dimostriamo che $tu$ si riduce in uno step a termini in $\can{S}$. Infatti
    poich\'e $t$ \`e normale, $tu$ si pu\`o ridurre solo a $t'u$ o a $tu'$ per
    opportuni termini $t'$ e $u'$. Ma il primo appartiene a $\can{S}$ perch\`e
    $t'\in\can{R}\rightarrow\can{S}$, e il secondo ci appartiene per ipotesi
    induttiva in quanto $\nu(u')<\nu(u)$. Per (CR3) su $\can{S}$ allora
    $tu\in\can{S}$. 
\end{proof}
\end{block}

\newcommand{\subrx}{\sub{\underline{\can{R}}}{\underline{X}}}
\begin{block} [Definizione]
    Sia $T[\underline{X}]$ un tipo con variabili libere in $\underline{X}$. Sia
    $\underline{U}$ un vettore di tipi della stessa lunghezza e siano $\can{R}$
    dei rispettivi candidati. Possiamo allora definire l'insieme
    $\red{T}\subrx$ di termini riducibili parametrici di tipo
    $\sub{\underline{U}}{\underline{X}}$ nel modo seguente:
    \begin{nlist} [1]
        \item  Se $T=X_i$ per qualche indice $i$, allora
            $\red{T}\subrx=\can{R}_i$.
        \item Se $T=V\rightarrow U$, allora $\red{T}\subrx = \red{V}\subrx
            \rightarrow \red{W}\subrx$.
        \item Se $T=\Pi Y.W$, allora $\red{T}\subrx$ \`e l'insieme dei termini
            $t$ di tipo $\sub{\underline{U}}{\underline{X}}$ tali che per ogni
            tipo $V$ e per ogni candidato $\can{S}$ di tale tipo vale che $tV
            \in \red{W}\subrx\sub{\can{S}}{Y}$.
    \end{nlist}
\end{block}

\begin{block} [Lemma]
    $\red{T}\subrx$ \`e un candidato di riducibilit\`a di tipo
    $\sub{\underline{U}}{\underline{X}}$.
\begin{proof}
    Lo facciamo per induzione sulla complessit\`a del tipo $T$. Il caso in cui
    $T$ \`e una variabile individuale, il teorema \`e una tautologia. Il caso in
    cui $T=V\rightarrow W$ lo abbiamo gi\`a fatto. Manca solo il caso in cui
    $T=\Pi Y.W$. 

    Verifichiamo (CR1). Sia $t\in \red{T}\subrx$, $V$ un tipo e $\can{S}$ un suo
    candidato. Allora $tV \in \red{W}\subrx\sub{\can{S}}{Y}$ per definizione.
    Usando l'ipotesi induttiva sul tipo $W$ si ha che $tV$ \`e fortemente
    normalizzabile. Ma vale anche che $\nu(t)<\nu(tV)$. Quindi anche $t$ \`e
    fortemente normalizzabile. 

    Per (CR2), supponiamo di avere $t\reduce t'$ con uno step di conversione.
    Allora, $tV\reduce t'V$, per cui $t'V \in \red{W}\subrx\sub{\can{S}}{Y}$ e
    quindi $t'\in \red{T}\subrx$.

    Infine, per (CR3), consideriamo $t$ un qualunque termine di tipo $T$
    neutrale. Supponiamo che per ogni $t'$ ottenuto dalla conversione di $t$ in
    un singolo step si abbia $t'$ riducibile parametrico. Allora per ogni tipo
    $V$ e relativo condidato $\can{S}$, le uniche conversioni di $tV$ sono della
    forma $tV\reduce t'V$. Usando l'ipotesi induttiva allora anche $tV$ \`e
    riducibile parametrico, e quindi si ha la tesi.
\end{proof}
\end{block}

\begin{block} [Lemmma]
    $\red{T\sub{V}{Y}} = \red{T}\subrx \sub{\red{V}\subrx}{Y}$.
\begin{proof}
    Come prima, facciamo un'induzione sulla complessit\`a del tipo $T$. Per
    comodit\`a, usiamo l'abbreviazione $A=\red{V}\subrx$.

    Iniziamo con il caso in cui $T=Z$ \`e una variabile individuale diversa da
    $Y$. Allora vale che 
    \[
        \red{T\sub{V}{Y}}\subrx = \red{Z}\subrx = \red{Z}\subrx\sub{A}{Y}.
    \]

    Se invece $T=Y$ si ha che
    \[
        \red{T\sub{V}{Y}}\subrx = \red{V}\subrx = \red{Y}\subrx\sub{
            \sub{V}\subrx}{Y}.
    \]

    Consideriamo ora il caso in cui $T=U\rightarrow W$. Vale che
    \begin{align*}
        \red{T\sub{V}{Y}}\subrx = 
        &\red{U\sub{V}{Y}\rightarrow W\sub{V}{Y}}\subrx =  \\
        &\red{U\sub{V}{Y}}\subrx \rightarrow \red{W\sub{V}{Y}}\subrx= \\
        &\red{U}\subrx\sub{A}{Y} \rightarrow\red{W}\subrx\sub{A}{Y} = \\
        &\red{U\rightarrow W}\subrx\sub{A}{Y}.
    \end{align*}

    Sia $Z$ come prima e svolgiamo il caso $T=\Pi Z. W$. Per definizione,
    $\red{\Pi Z. W\sub{V}{Y}}\subrx$ \`e l'insieme di tutti i termini $t$ per
    cui per ogni tipo $U$ e relativo candidato $\can{S}$ vale che
    \[
        tU\in \red{W\sub{V}{Y}}\subrx\sub{\can{S}}{Z} = 
        \red{W}\subrx\sub{\can{S}}{Z}\sub{A}{Y}.
    \]
    Dunque si ottiene la tesi per la definizione di $\red{\Pi Z.W}$.

    Infine il caso in cui $T=\Pi Y.W$ \`e semplice perch\`e $Y$ non occorre
    libera in $T$.
\end{proof}
\end{block}

\begin{block} [Lemma]
    Se per ogni tipo $V$ e per ogni candidato di riducibilit\`a $\can{S}$ per
    $V$ vale che $w\sub{V}{Y}\in\red{W}\subrx\sub{\can{S}}{Y}$, allora $\Lambda
    Y. w\in\red{\Pi Y.W}\subrx$.
\begin{proof}
    Dimostriamo per induzione su $\nu(w)$ che tutte le conversioni in uno step
    di $(\Lambda Y.w)V$ sono in $\red{W} \subrx\sub{\can{S}}{Y}$. Una
    conversione di tali conversioni possono essere soltanto di due forme.  La
    prima \`e $(\Lambda Y.w')V$, con $w'$ una conversione di $w$. Ma allora
    $\nu(w')<\nu(w)$ e si usa l'ipotesi induttiva. La seconda forma \`e del tipo
    $w\sub{V}{Y}$, e questa \`e riducibile parametrico per ipotesi del lemma. 

    Allora la dimstrazione si conclude per (CR3).
\end{proof}
\end{block}

\begin{block} [Lemma]
    Se $t\in \red{\Pi Y.W}\subrx$, allora $tV\in \red{W\sub{V}{Y}}\subrx$ per
    ogni tipo $V$.
\begin{proof}
    Per la definizione di $\red{\Pi Y.W}$, per ogni candidato $\can{S}$ per $V$
    vale che $tV\in \red{W}\subrx\sub{\can{S}}{Y}$. Allora vale anche per
    $\can{S} = \red{V}\subrx$ e la tesi segue per il lemma XXX.
    % @todo sistemare il riferimento
\end{proof}
\end{block}

\begin{block} [Lemma]
    Se per ogni $u\in \red{U}\subrx$ vale che $v\sub{u}{x} \in \red{V}\subrx$,
    allora $\lambda x^U.v\in \red{U\rightarrow V}\subrx$.
\begin{proof}
    Dimostriamo per induzione su $\nu(u)+\nu(v)$ che tutte le conversioni di
    $(\lambda x^U.v)u$ sono riducibili parametrici. Infatti tale termine si
    converte in $(\lambda x^U.v)u'$, con $u'$ conversione di $u$, oppure in
    $(\lambda x^U.v')u$ con $v'$ conversione di $v$, oppure in $v\sub{u}{x}$. I
    primi due casi si risolvono con l'ipotesi induttiva, il terzo con l'ipotesi
    del lemma.

    Infine il teorema si dimostra per la propriet\`a (CR3).
\end{proof}
\end{block}

\begin{block} [Definizione]
    Un termine $t$ di tipo $T$ \`e riducibile se \`e in
    $\red{T}\sub{\underline{\can{SN}}}{\underline{X}}$ dove $X_1,\dots,X_m$ sono
    le variabili libere di $T$ e $\underline{\can{SN}}_i$ \`e l'insieme dei
    termini fortemente normalizzabili di tipo $X_i$.
\end{block}

\begin{block} [Proposizione]
    Sia $t$ un termine di tipo $T$ le cui variabili libere sono $x_1, \dots,
    x_n$ di tipo rispettivamente $U_1, \dots, U_n$. Supponiamo che le variabili
    libere dei tipi $T$ e di tutti gli $U_i$ siano $X_1, \dots, X_m$. Siano
    $\can{R}_1, \dots, \can{R}_m$ candidati di riducibilit\`a per dei tipi $V_1,
    \dots, V_m$ e siano inoltre $u_1, \dots, u_n$ termini di tipo
    $U_1\sub{\underline{V}}{\underline{X}}, \dots
    U_n\sub{\underline{V}}{\underline{X}}$ presi nei rispettivi
    $\red{U_i}\subrx$. Allora $t\sub{\underline{V}}{\underline{X}}
    \sub{\underline{u}}{\underline{x}} \in \red{T}\subrx$.
\newcommand{\subvx}{\sub{\underline{V}}{\underline{X}}}
\newcommand{\subut}{\sub{\underline{u}}{\underline{x}}}
\begin{proof}
    Per induzione sulla complessit\`a di $t$. Distinguiamo allora i seguenti
    casi:
    \begin{nlist}[i]
        \item $t=x_i$. Questo caso \`e una tautologia.
        \item $t=wv$, con $w$ di tipo $W\rightarrow T$ e $v$ di tipo $W$. Per
            ipostesi induttiva vale che $w\subvx\subut\in \red{W \rightarrow
            T}\subrx$ e che $v\subvx\subut \in \red{W}\subrx$. In questo caso la
            tesi segue dalla definizione di $\red{W\rightarrow T}$.
        \item $t=wS$. Questo caso \`e una diretta conseguenza del lemma XXX
            sull'istanziazione. 
            % @todo
        \item $t=\Lambda Z. Y$. Questo discende dal lemma XXX sulla
            generalizzazione.
        \item $t=\lambda y^P.w$. Questo caso si fa con il lemma XXX sui tipi
            freccia.
    \end{nlist}
\end{proof}
\end{block}

\begin{block} [Proposizione]
    Tutti i termini del sistema F sono riducibili.
\begin{proof}
    Basta usare la proposizione precedente e prendere $\can{R}_i=\can{SN}_i$ e
    $u_i=x_i$.
\end{proof}
\end{block}

\begin{block} [Teorema]
    Tutti i termini del sistema F sono fortemente normalizzabili.
\end{block}

\end{document}
