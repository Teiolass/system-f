\documentclass{beamer}

\usepackage[utf8]{inputenc}
\usepackage[italian]{babel}
\usepackage{amssymb}
\usepackage{enumerate}
\usepackage{amsmath}
\usepackage{nicematrix}
\usetheme{Copenhagen}
\beamertemplatenavigationsymbolsempty
\setbeamertemplate{footline}[frame number]

%%%%%%%%%%%%%%%%%%%%%%%%%%%%%%%%%%%%%%%%%%%%%%%%%%%%%%%%%%%%%%%%%%%%%%%%%%%%%%%%%%%%%%%%%%%%%%%%%%
\usepackage[many]{tcolorbox}
\definecolor{myblue1}{RGB}{129, 47, 51}
\definecolor{myblue2}{RGB}{129, 47, 51}
\definecolor{myblue3}{RGB}{129, 47, 51}
\definecolor{myblue4}{RGB}{129, 47, 51}
% \definecolor{myblue2}{RGB}{129, 47, 51}{112, 94, 120}
% \definecolor{myblue3}{RGB}{129, 47, 51}
% \definecolor{myblue4}{RGB}{129, 47, 51}

\setbeamercolor*{structure}{fg=myblue1,bg=blue}
\setbeamercolor*{palette primary}{use=structure,fg=white,bg=structure.fg}
\setbeamercolor*{palette secondary}{use=structure,fg=white,bg=structure.fg!75!black}
\setbeamercolor*{palette tertiary}{use=structure,fg=white,bg=structure.fg!50!black}
\setbeamercolor*{palette quaternary}{fg=black,bg=white}

\setbeamercolor*{block title example}{fg=white,bg=myblue4}

\setbeamertemplate{blocks}[rounded][shadow=false]

\makeatletter
\defbeamertemplate*{frametitle}{mydefault}[1][left]
{
  \ifbeamercolorempty[bg]{frametitle}{}{\nointerlineskip}%
  \nointerlineskip%
  \@tempdima=\textwidth%
  \advance\@tempdima by\beamer@leftmargin%
  \advance\@tempdima by\beamer@rightmargin%
  \begin{tcolorbox}[
  enhanced,
  outer arc=0pt,
  arc=0pt,
  boxrule=0pt,
  top=4pt,
  bottom=3pt,
  enlarge left by=-\beamer@leftmargin,
  enlarge right by=-\beamer@rightmargin,
  width=\paperwidth,
  height=1cm,
  nobeforeafter,
  interior style={
    left color=myblue2,
    right color=myblue2
    },
  % shadow={0mm}{-0.4mm}{0mm}{black!60,opacity=0.6},    
  % shadow={0mm}{-0.8mm}{0mm}{black!40,opacity=0.4},    
  ]
      \usebeamerfont{frametitle}\usebeamercolor[fg]{white}%
    \vbox{}\vskip-1ex%
    \if@tempswa\else\csname beamer@fte#1\endcsname\fi%
    \insertframetitle\par%
    {%
      \ifx\insertframesubtitle\@empty%
      \else%
      {\usebeamerfont{framesubtitle}\usebeamercolor[fg]{framesubtitle}\insertframesubtitle\strut\par}%
      \fi
    }%
    \vskip-1ex%
    \if@tempswa\else\vskip-.3cm\fi% set inside beamercolorbox... evil here...
  \end{tcolorbox}%
}
\defbeamertemplate*{footline}{mysplit theme}
{%
  \leavevmode%
  \hbox{\begin{beamercolorbox}[wd=.5\paperwidth,ht=2.5ex,dp=1.125ex,leftskip=.3cm plus1fill,rightskip=.3cm]{author in head/foot}%
    \usebeamerfont{author in head/foot}\insertshortauthor
  \end{beamercolorbox}%
  \begin{beamercolorbox}[wd=.5\paperwidth,ht=2.5ex,dp=1.125ex,leftskip=.3cm,rightskip=.3cm plus1fil]{title in head/foot}%
    \usebeamerfont{title in head/foot}\insertshorttitle\hfill
    \insertframenumber/\inserttotalframenumber\hspace*{0.5em}
  \end{beamercolorbox}}%
  \vskip0pt%
}
\makeatother
\setbeamertemplate{itemize items}[circle]
\setbeamercolor{enumerate item}{fg=black}
\setbeamertemplate{enumerate items}[default]
% /* Color Theme Swatches in RGBA */
% .evening-on-the-beach-1-rgba { color: rgba(242, 253, 175, 1); }
% .evening-on-the-beach-2-rgba { color: rgba(253, 163, 66, 1); }
% .evening-on-the-beach-3-rgba { color: rgba(112, 93, 119, 1); }
% .evening-on-the-beach-4-rgba { color: rgba(165, 170, 163, 1); }
% .evening-on-the-beach-5-rgba { color: rgba(128, 47, 51, 1); }
%%%%%%%%%%%%%%%%%%%%%%%%%%%%%%%%%%%%%%%%%%%%%%%%%%%%%%%%%%%%%%%%%%%%%%%%%%%%%%%%%%%%%%%%%%%%%%%%%%


\renewcommand{\l}{\lambda}
\renewcommand{\L}{\Lambda}
\newcommand{\conv}{\rightsquigarrow}
\newcommand{\ar}{\rightarrow}
\newcommand{\can}{\mathcal}

\title[Lambda calcolo]{Lambda calcolo del Secondo Ordine}
\author{Alessio Marchetti}
\date{}


\begin{document}

\frame{\titlepage}

\begin{frame}
    \frametitle{Il $\lambda$-calcolo} I $\lambda$-termini sono parole
    sull'alfabeto costituito da variabili $x_0, x_1, \dots$, dall'astrattore
    $\lambda$ e dalle parentesi $(,)$. 
\\~\

    Definiamo induttivamente il loro insieme $\Lambda$ come:
    \begin{itemize}
        \item $x_i \in \Lambda$ per tutte le variabili $x_i$.
        \item $t\in \Lambda\ \Rightarrow\ (\lambda x_i.t)\in\Lambda$ per tutte
            le variabili $x_i$.
        \item $t,u\in \Lambda\ \Rightarrow\ (tu)\in \Lambda$.
    \end{itemize}
\end{frame}

\begin{frame}
    \frametitle{Semplificare la notazione}
    L'astrazione si associa a destra:
    \[
        \lambda x_1\cdots x_n.t = \lambda x_1. \cdots \lambda x_n.t = (\lambda
        x_1.(\lambda x_2.\cdots (\lambda x_n.(t))\cdots)
    \]
    \\~\

    L'applicazione si associa a sinistra:
    \[
        t_1\cdots t_n = (\cdots(t_1)\cdots t_n) 
    \]
    \\~\
    
    Un sottotermine di $t$ \`e un qualunque termine $u$ utilizzato nella
    costruzione induttiva di $t$. 

    Esempio: $xy$ \`e un sottotermine di $\lambda x. xyz$.
\end{frame}

\begin{frame}
    \frametitle{Riduzione}
    Un termine $t$ si converte a un termine $t'$ se sono nella forma
    \[
        t = (\lambda x. u)v \qquad t' = u[v/x]
    \]
    dove l'espressione $u[v/x]$ indica la sostituzione di $v$ nella variabile
    $u$. Si dice che $t'$ \`e una conversione di $t$.
    \\~\

    Un termine $u$ si riduce a un termine $u'$ se esiste una successione di
    passi $u = u_1 = \cdots = u_n = u'$ in cui ogni $u_{i+1}$ \`e ottenuto
    sostituendo un sottotermine di $u_i$ con una sua conversione. In questo caso
    si scrive $u \conv u'$.
\end{frame}

\begin{frame}
    \frametitle{Esempio di Riduzione}
    \begin{align*}
        (\l x.xx)(\l y.y)z \conv (\l y.y)(\l y.y)z 
        \conv (\l y.y)z \conv z.
    \end{align*}
    \\~\

    $z$ non pu\`o essere ulteriormente ridotto, si dice che \`e in forma
    normale.
    \\~\

    \[
        \Omega = (\l x.xx)(\l x.xx) \conv (\l x.xx)(\l x.xx) \conv\cdots
    \]
    $\Omega$ non ha una forma normale.
\end{frame}

\begin{frame}
    \frametitle{I numeri naturali}

    Si possono rappresentare i numeri naturali con termini del tipo
    \[
        \bar{n} = \l s. \l z. \underbrace{s ( s ( \cdots (s}_{n\text{
            ripetizioni}} (z))\cdots )).
    \]
    \\~\
    $\bar{n}$ \`e il funzionale che compone $n$ volte la funzione data in input.
    \[
        \bar{3} f x \conv f(f(f(x))) = f^3x
    \]
    \\~\

    Questi numerali sono in forma normale.
\end{frame}

\begin{frame}
    \frametitle{Funzioni rappresentabili}

    Un lambda termine $E$ rappresenta un una funzione parziale ${f\colon
    \mathbb{N}\rightarrow \mathbb{N}\cup \{\bot\} }$ se per ogni intero $n$
    \[
        E\bar{n}\conv \overline{m}
    \]
    se e solo se $f(n)=m$.
\end{frame}

\begin{frame}
    \frametitle{Alcuni esempi} 

    \[
        A = \l p. \l q. \l f. \l x. (pf)(qfx)
    \]
    \begin{align*}
        A \bar{3} \bar{2} \conv \l f. \l x. (\bar{3}f)(\bar{2}fx)
        \conv \l f. \l x. f^3(f^2x) = \bar{5}.
    \end{align*}
    \\~\

    \[
        M = \l p. \l q. \l f. \l x. q(pf)x
    \]
    \begin{align*}
        M \bar{3} \bar{2} \conv \l f. \l x. \bar{2} (\bar{3}f)x
        \conv \l f. \l x. (\bar{3}f)(\bar{3}f)x \conv \l f. \l x.f^3f^3x =
        \bar{6}.
    \end{align*}
    \\~\

    \[
        E = \l p. \l q. \l f. \l x. qpfx
    \]
    \begin{align*}
        E \bar{3} \bar{2} \conv \l f. \l x. \bar{2} \bar{3}fx
        \conv \l f. \l x. \bar{3}\bar{3}fx \conv \l f. \l x. \bar{3}^3 fx =
        \bar{9}.
    \end{align*}
\end{frame}

\begin{frame}
    \frametitle{Due riduzioni ``patologiche''}
    Un termine che non diventa pi\`u semplice:
    \[
        \omega_3\omega_3 = (\l x. xxx)(\l x. xxx) \conv (\l x. xxx)(\l x.
        xxx)(\l x. xxx) \conv \cdots
    \]
    \\~\
    
    Un termine normalizzante ma non fortemente normalizzante:
    \[
        (\l x. y)(\omega_3\omega_3) \conv y
    \]
    \[
        (\l x. y)(\omega_3\omega_3) \conv (\l x. y)(\omega_3\omega_3\omega_3)
        \conv \cdots
    \]
\end{frame}

\begin{frame}
    \frametitle{Lambda calcolo tipato semplice (1)}
    Idea: aggiungere vincoli sul come applicare un termine ad un altro.
    \\~\
    
    Si definiscono induttivamente i tipi come:
    \begin{itemize}
        \item I tipi variabile $U_1, U_2, \dots$ sono tipi.
        \item Se $U$ e $V$ sono tipi $(U\ar V)$ \`e un tipo.
    \\~\
    \end{itemize}

    Per comodit\`a $\ar$ si associa a destra:
    \[
        U_1\ar U_2\ar \cdots \ar U_n = (U_1 \ar ( U_2 \ar ( \cdots \ar (U_n)
        \cdots ))).
    \]
\end{frame}


\begin{frame}
    \frametitle{Lambda calcolo tipato semplice (2)}
    
    I termini si definiscono come:
    \begin{itemize}
        \item Le variabili di tipo $U$ sono $x_1, x_2, \dots$ per ogni tipo $U$.
        \item Se $x$ \`e una variabile di tipo $U$ e $t$ \`e un termine di tipo
            $V$, allora $\l x. t$ \`e un termine di tipo $U\ar V$.
        \item Se $f$ \`e un termine di tipo $U\ar V$ e $t$ \`e un termine di
            tipo $U$, allora $ft$ \`e un termine di tipo $V$.
    \\~\
    \end{itemize}

    Per indicare che un termine $t$ \`e di tipo $U$ si scrive anche $t^U$.
\end{frame}

\begin{frame}
    \frametitle{Numerali}
    Sia $U$ un tipo. Si definiscono i numerali come i termini
    \[
        \bar{n} = \l s^{U\ar U}. \l z^{U}. \underbrace{s ( s ( \cdots (s}_{n\text{
            ripetizioni}} (z))\cdots )).
    \]
    \\~\
    La definizione \`e uguale alla precedente, ma con le variabili tipate.
\end{frame}


\begin{frame}
    \frametitle{Funzioni rappresentabili}

    Sia $I$ il tipo $(U\ar U)\ar (U\ar U)$.
    Somma e moltiplicazione come definite prima si possono tipare:
    \[
        A = \l p^I. \l q^I. \l f^{U\ar U}. \l x^U. (pf)(qfx)
    \]
    \[
            M = \l p^I. \l q^I. \l f^{U\ar U}. \l x^U. q(pf)x
    \]
    \\~\
    
    L'esponenziazione, che era stata precedente definita come
    \[
        E = \l p. \l q. \l f. \l x. qpfx
    \]
    non si pu\`o tipare. Le funzioni rappresentabili in questo calcolo sono
    esattamente le funzioni polinomiali a tratti.
\end{frame}

\begin{frame} 
    \frametitle{Lambda calcolo del Secondo Ordine (1)}
    Si definiscono induttivamente i tipi come:
    \begin{itemize}
        \item I tipi variabile $U_1, U_2, \dots$ sono tipi.
        \item Se $U$ e $V$ sono tipi $(U\ar V)$ \`e un tipo.
        \item Se $X$ \`e un tipo variabile e $U$ \`e un tipo, allora $(\Pi X.U)$
            \`e un tipo.
    \\~\
    \end{itemize}
\end{frame} 

\begin{frame} 
    \frametitle{Lambda calcolo del Secondo Ordine (2)}
    I termini si definiscono come:
    \begin{itemize}
        \item Le variabili di tipo $U$ sono $x_1, x_2, \dots$ per ogni tipo $U$.
        \item Se $x$ \`e una variabile di tipo $U$ e $t$ \`e un termine di tipo
            $V$, allora $\l x. t$ \`e un termine di tipo $U\ar V$.
        \item Se $f$ \`e un termine di tipo $U\ar V$ e $t$ \`e un termine di
            tipo $U$, allora $ft$ \`e un termine di tipo $V$.
        \item Se $t$ \`e un termine di tipo $\Pi X. U$ e $V$ \`e un tipo, allora
            $tV$ \`e un termine di tipo $U[V/X]$.
        \item Se $t$ \`e un termine di tipo $U$ e $X$ \`e una variabile tipo che
            non compare libera nei tipi delle variabili libere di $t$, allora
            $\L X. t$ \`e un termine di tipo $\Pi X. U$.
    \end{itemize}
\end{frame} 

\begin{frame}
    \frametitle{Numerali}
    Definiamo il tipo intero
    \[
        N = \Pi X. (X\ar X)\ar (X\ar X).
    \]
    I numerali hanno la forma
    \[
        \bar{n} = \L X. \l f^{X\ar X}. \l x^X. f^nx.
    \]
\end{frame}

\begin{frame}
    \frametitle{L'esponenziale}
    In questo caso si pu\`o definire l'esponenziale:
    \[
        E = \l p^N. \l q^N \L X. \l f^{X\ar X}. \l x^X. q(X\ar X) (pX) fx. 
    \]
    \begin{align*}
        E\bar{2}\bar{3} \conv &\L X. \l f^{X\ar X}. \l x^X. \bar{3}(X\ar X)
            (\bar{2}X) f \conv \\
        &\L X. \l f^{X\ar X}. \l x^X. (\l g^{(X\ar X)\ar (X\ar X)}. \l y^{X\ar X}
            . g^3 y) (\bar{2}X) fx \conv \\
        &\L X. \l f^{X\ar X}. \l x^X.  (\bar{2}X)^3 fx \conv \\
        &\L X. \l f^{X\ar X}.  \l x^X. (\l g^{X\ar X}.\l y^X. g^2 y)^3 fx \conv
            \\
        &\L X. \l f^{X\ar X}. \l x^X. (\l g^{X\ar X}.\l y^X. g^2 y)^2 f^2 x \conv
            \\
        &\L X. \l f^{X\ar X}. \l x^X. (\l g^{X\ar X}.\l y^X. g^2 y) f^4 x \conv
            \\
        &\L X. \l f^{X\ar X}. \l x^X. f^8 x \conv \bar{8}
    \end{align*}
\end{frame}

\begin{frame}
    \frametitle{Normalizzazione forte}
    Un calcolo \`e fortemente normalizzante quando tutti i suoi termini sono
    fortemente normalizzanti, ovvero ogni percorso di riduzione porta a una
    forma normale.
    \\~\

    Il lambda calcolo tipato semplice e tipato al secondo ordine sono fortemente
    normalizzanti.
\end{frame}

\begin{frame}
    \frametitle{Normalizzazione forte per il tipato semplice}
    Definiamo la riducibilit\`a:
    \begin{itemize}
    \item I riducibili di tipo variabile $U$ sono i termini di tipo $U$
        fortemente normalizzanti. 
    \item I riducibili di tipo $U\ar V$, per $U$, $V$ tipi qualunque, sono i
        termini $f$ di tipo $U\ar V$ tali che per ogni $t$ riducibile di tipo
            $U$, $ft$ \`e riducibile di tipo $V$.
    \end{itemize}
\end{frame}

\begin{frame}
    \frametitle{Propriet\`a dei riducibili}
    \begin{enumerate}
        \item I riducibili sono fortemente normalizzanti.
        \item Se $t$ \`e riducibile e $t\conv u$, allora anche $u$ \`e
            riducibile.
        \item Sia $t$ una variabile oppure un termine nella forma $uv$. Se ogni
            riduzione di $t$ in uno step porta a un termine riducibile, allora
            $t$ \`e riducibile.
        \\~\
    \end{enumerate}
    Si dimostra per induzione sulla complessit\`a dei termini che tutti i
    termini sono riducibili.
\end{frame}


\begin{frame}
    \frametitle{Un tentativo di riadattamento}
    Si vuole provare a estendere il concetto di riducibilit\`a anche per i
    termini del secondo ordine.
    \\~\

    ``Un termine $t$ di tipo $\Pi X. U$ \`e riducibile se per ogni tipo $V$,
    $tV$ \`e riducibile di tipo $U[V/X]$.''
    \\~\

    Definizione impredicativa: per esempio $V = \Pi X. U$.
\end{frame}

\begin{frame}
    \frametitle{Candidati di riducibilit\`a}
    Un candidato di riducibilit\`a per un tipo $U$ \`e un insieme $\can{R}$ per
    cui valgono:
    \begin{enumerate}
        \item Se $t\in \can{R}$, allora $t$ \`e fortemente normalizzante.
        \item Se $t\in \can{R}$ e $t\conv u$, allora $u\in\can{R}$.
        \item Sia $t$ un termine di tipo $U$ che sia una variabile oppure
            un'applicazione nella forma $uv$ o $uU$. Se ogni conversione in un
            passo di $t$ porta a un termine in $\can{R}$, allora $t\in\can{R}$.
    \end{enumerate}
\end{frame}

\newcommand{\red}{\text{RED}}
\newcommand{\ronx}{[\underline{\can{R}}/\underline{X}]}

\begin{frame}
    \frametitle{Riducibilit\`a parametrica}
    Sia $T$ un tipo con variabili tipo libere in $\underline{X}$.

    Siano $\underline{U}$ dei tipi e $\underline{\can{R}}$ rispettivi candidati.
    \\~\

    Definiamo i riducibili parametrici $\red_T\ronx$ di tipo $T[\underline{U}/
    \underline{X}]$ come:
    \begin{itemize}
        \item Se $T=X_i$, allora $\red_T\ronx = \can{R}_i$.
        \item Se $T=V\ar W$, allora $f\in\red_T\ronx$ se per ogni
            $t\in\red_V\ronx$, vale che $ft\in\red_W\ronx$.
        \item Se $T=\Pi Y.V$, allora $t\in\red_T\ronx$ se per ogni tipo $W$ e
            per ogni candidato $\can{S}$ per $W$ vale che $tW\in\red_{V[W/Y]}
            [\underline{\can{R}}/\underline{X}, \can{S}/Y]$.
    \\~\
    \end{itemize}
    
    I riducibili parametrici sono candidati di riducibilit\`a.
\end{frame}

\begin{frame}
    \frametitle{Normalizzazione per il secondo ordine}
    A questo punto si possono definire i riducibili di tipo $T$ come i termini
    in $\red_T[\underline{\can{SN}}/\underline{X}]$, dove gli $X_i$ sono i tipi
    delle variabili tipo libere di $T$ e gli $\can{SN}_i$ sono i termini
    fortemente normalizzanti di tipo $X_i$.
    \\~\

    Si dimostra poi che tutti i termini sono riducibili, e quindi fortemente
    normalizzanti.
\end{frame}
\end{document}
